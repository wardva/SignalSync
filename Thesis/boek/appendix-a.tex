\chapter{Resultaten: DTW experiment}
\label{appendix-a}

Met dit experiment werd getracht de nauwkeurigheid van het DTW algoritme te bepalen wanneer streams gebufferd worden. Hiertoe werd eerst de latency tussen twee audiofragmenten bepaald. Vervolgens werd iteratief de duur van het fragment met 10 seconden verkleind waarna het algoritme opnieuw werd uitgevoerd. Tenslotte werden de buffergrootte en nauwkeurigheid van de resultaten vergeleken.

Er werden twee audiofragmenten gebruikt met een latency van $2.390s$. Beide fragmenten hebben samplefrequentie van $8000Hz$. Eén van de twee fragmenten is een opname van het origineel en bijgevolg van matige kwaliteit.

Het experiment is uitgevoerd in \textit{Sonic Visualiser} met behulp van de \textit{Match Performance Aligner} plug-in. Deze plug-in laat synchronisatie toe met behulp van het DTW algoritme. De implementatie wordt besproken in artikel \cite{dixon2005match}. Voor dit experiment zijn de default instellingen gebruikt. De plug-in bepaalt elke twintig milliseconden de latency tussen beide fragmenten.

De volgende tabel geeft de resultaten van het experiment weer. De eerste kolom bevat de lengte van de vergeleken fragmenten in seconden. Deze lengte stelt de buffergrootte voor van een audiostream. De tweede kolom geeft aan hoeveel seconden van de stream moet worden verwerkt tot er een stabiel resultaat wordt bekomen. De derde kolom geeft het gemiddelde weer van de gevonden latencies. Deze waarde wordt berekend vanaf het algoritme een stabiel resultaat heeft gevonden. De vierde kolom bevat de standaardafwijking van dit resultaat.\\

\begin{center}
\begin{tabular}{ c  c  c  c }
	\hline
	\textbf{Lengte} & \textbf{Tijd tot stabiel} & \textbf{Gemiddelde latency} & \textbf{Standaardafwijking} \\
	\hline
	60s & 2.540s & 2,393s & 0.048s \\
	50s & 2.540s & 2,390s & 0.095s \\
	40s & 2.540s & 2,394s & 0.020s \\
	30s & 2.540s & 2,384s & 0.145s \\
	20s & 2.540s & 2,390s & 0.108s \\
	10s & 2.540s & 2,395s & 0.025s \\
	\\
\end{tabular}\\
\end{center}

Uit bovenstaande resultaten kunnen verschillende zaken geconcludeerd worden. De standaardafwijking toont aan dat de individuele resultaten (die iedere 20ms gegenereerd worden) niet nauwkeurig genoeg zijn voor het gestelde probleem. De gemiddelde waarde komt wel in de buurt van de werkelijke latency maar is nog steeds niet zo nauwkeurig. Ook moet bij de berekening van het gemiddelde rekening gehouden worden met het feit dat het algoritme pas na een bepaalde tijd een stabiel resultaat vindt, in dit geval 2.540s.
