\chapter{Ontwerp van de softwarebibliotheek}
\label{appendix-b}

In deze bijlage zal het ontwerp van de softwarebibliotheek besproken worden. Het ontwerp van de verschillende soorten streams zal niet worden herhaald aangezien deze materie al eerder behandeld is in sectie \ref{streams-design}.

\section*{Bufferen van streams}

Alle klassen die te maken hebben met het bufferen van streams (het opsplitsen in zogenaamde \textit{slices}) bevinden zich in het package \texttt{be.signalsync.slicer}. Het hoe en waarom van deze verwerking werd al uitgebreid behandeld in sectie \ref{streambuffers}.

\subsubsection{De abstracte klasse Slicer}

\begin{figure}[h!]
	\captionsetup{width=0.7\textwidth}
	\caption{UML diagram de observer logica van de klasse \texttt{Slicer}}
	\begin{center}
		\advance\parskip0.3cm
		\begin {tikzpicture}
	\begin{package}[SlicerPackage]{be.signalsync.slicer}
		
		\begin{abstractclass}[template parameter=<T>, text width=10.5cm]{Slicer}{0,0}
		
			\operation {+ addEventListener(listener : SliceListener) : void}
			\operation {+ removeEventListener(listener : SliceListener) : void}
			
			\operation {\# emitSliceEvent(result : T, beginTime : double) : void}
			\operation {\# emitDoneEvent() : void}
		\end{abstractclass}
		
		\begin{interface}[template parameter=<T>, text width=7.5cm]{SliceListener}{0,-7}
			\operation {done(s : Slicer\textless T\textgreater) : void}
			\operation {onSliceEvent(e : SliceEvent\textless T\textgreater) : void}
		\end{interface}
		
		\unidirectionalAssociation{Slicer}{\hspace{1.7cm}listeners}{\hspace{1cm}0..*}{SliceListener}
		
		\begin{class}[template parameter=<T>, text width=12cm]{SliceEvent}{0,-11}
			\attribute {- slice : T}
			\attribute {- slicer : StreamSlicer\textless T\textgreater}
			\attribute {- timestamp : double}
			
			\operation {+ SliceEvent(slicer : Slicer\textless T\textgreater, slices : T, beginTime : double)}
			\operation {// getters}
		\end{class}

	\end{package}
\end {tikzpicture}

	\end{center}
	\label{slicerObserver}
\end{figure}

Een \texttt{Slicer} is een abstracte klasse die een stream of verzameling van streams inleest, in stukjes knipt en vervolgens deze stukjes teruggeeft aan elke geïnteresseerd object. Hoe en wat er precies in stukjes geknipt wordt hangt af van het subtype van \texttt{Slicer}. De klasse \texttt{Slicer} is enkel verantwoordelijk voor het registreren, bijhouden en verwittigen van objecten die geïnteresseerd zijn in slices. Dit proces wordt verwezenlijk met behulp van het \textit{observer} ontwerppatroon (uitgebreid besproken in boek \cite{vlissides1995design}). 

Figuur \ref{slicerObserver} toont de drie klassen die een rol spelen het observer mechanisme. Elke klasse is generiek en maakt gebruik van type parameter \texttt{T}. Dit is het type van hoe een slice wordt voorgesteld. Bij een slice van een stream is dit type bijvoorbeeld een array van \texttt{float} waarden.

Zoals in de figuur te zien is bevat \texttt{Slicer} een verzameling van geïnteresseerde objecten. De klasse van een geïnteresseerd object moet de interface \texttt{SliceListener} implementeren. Wanneer een object geregistreerd is kan het \texttt{SliceEvent} objecten ontvangen. Dit object bevat de laatste nieuwe slice, de \texttt{Slicer} vanwaar de slice afkomstig is en een timestamp. Wanneer er geen slices meer verzameld kunnen worden kan de methode \texttt{done} van elke geïnteresseerd object worden opgeroepen.

\subsubsection{StreamSlicer}

\begin{figure}[h!]
	\captionsetup{width=0.7\textwidth}
	\caption{UML diagram van de klasse \texttt{StreamSlicer} en haar supertypes.}
	\begin{center}
		\advance\parskip0.3cm
		\input{figuren/streamSlicerUML}
	\end{center}
	\label{streamSlicer}
\end{figure}

Een \texttt{StreamSlicer} is een subklasse van \texttt{Slicer} die de data afkomstig van een stream opdeelt in \texttt{float} arrays van een welbepaalde lengte. Om toegang te krijgen tot de data van een \texttt{Stream} implementeert de \texttt{StreamSlicer} de interface \texttt{StreamProcessor}. Via \texttt{process} krijgt deze klasse opeenvolgende \texttt{float} arrays met samples binnen die worden opgeslagen in buffers. Wanneer er een slice gereed is worden de buffers samengevoegd tot één float array en wordt \texttt{emitSliceEvent} van de superklasse \texttt{Slicer} opgeroepen. Wanneer \texttt{processingFinished} wordt opgeroepen en de stream dus geen data meer beschikbaar heeft, dan wordt \texttt{emitDoneEvent} opgeroepen zodat alle \texttt{StreamListeners} hiervan op de hoogte zijn.

Figuur \ref{streamSlicer} toont een UML diagram met de besproken klassen en interfaces. Sommige low-level methodes en attributen zijn hierbij weggelaten. 

\subsubsection{StreamSetSlicer}

\begin{figure}[h!]
	\captionsetup{width=0.7\textwidth}
	\caption{UML diagram van de klasse \texttt{StreamSetSlicer} en haar supertypes.}
	\begin{center}
		\advance\parskip0.3cm
		\input{figuren/streamSetSlicerUML}
	\end{center}
	\label{streamSetSlicer}
\end{figure}

Een \texttt{StreamSetSlicer} is een subtype van de klasse \texttt{Slicer} die verantwoordelijk is voor het slicen van alle \textbf{audiostreams} van een \texttt{StreamSet}. Per \texttt{StreamGroup} wordt er dus maar één stream in stukjes verdeeld. Dit is logisch aangezien de synchronisatiealgoritmes voor het bepalen van de latency enkel worden uitgevoerd op slices van de audiostream.

Bij de creatie van een \texttt{StreamSetSlicer} wordt van elke \texttt{StreamGroup} uit de \texttt{StreamSet} de audiostream opgehaald. Op dit \texttt{Stream} object wordt vervolgens de methode \texttt{createSlicer} opgeroepen waarop er een \texttt{StreamSlicer} wordt teruggegeven. De \texttt{StreamSetSlicer} voegt zichzelf als geïnteresseerd object aan de \texttt{StreamSlicer} toe. Ook wordt de \texttt{StreamSlicer}
gekoppeld aan de \texttt{StreamGroup} door ze als sleutel en waarde toe te voegen aan een \texttt{Map}.

Na afloop van deze initialisatie wordt er gewacht tot wanneer alle streams waarop de \texttt{StreamSetSlicer} zich geregistreerd heeft een \texttt{SliceEvent} (met een \texttt{float} array) verstuurd hebben. Al deze slices worden vervolgens samen met hun corresponderende \texttt{StreamGroup} via de \texttt{emitSliceEvent} methode verstuurd naar de geïnteresseerde objecten. Dit proces wordt herhaald tot er geen enkele \texttt{Stream} nog data ter beschikking heeft.

Figuur \ref{streamSetSlicer} toont een vrij abstract klassendiagram van de \texttt{StreamSetSlicer} en haar supertypen. Low-level attributen zoals de buffers, threadpools en locks zijn hierop weggelaten.

\section*{Oproepen van de algoritmes}

\begin{figure}[h!]
	\captionsetup{width=0.7\textwidth}
	\caption{UML diagram van de klassen met synchronisatiealgoritmes.}
	\begin{center}
		\advance\parskip0.3cm
		\begin {tikzpicture}[scale=0.7]
\tikzstyle{every node}=[font=\scriptsize]
	\begin{package}[StreamPackage]{be.signalsync.syncstrategy}
		\setul{1pt}{.4pt}
		\begin {abstractclass}[text width=8cm]{SyncStrategy}{0,0}
			\operation{+ \ul{createDefault}() : SyncStrategy}
			\operation{+ \ul{createFingerprintStrategy}() : FingerprintSyncStrategy}
			\operation{+ \ul{createCrossCovariance}() : CrossCovarianceSyncStrategy}
			\operation[0]{+ findLatencies(sliceSet : List\textless float[]\textgreater) : List\textless Double\textgreater}
		\end {abstractclass}
		
		\begin {class}[text width=8cm]{FingerprintSyncStrategy}{-1,-6}
			\inherit{SyncStrategy}
			\operation{+ findLatencies(sliceSet : List\textless float[]\textgreater) : List\textless Double\textgreater}
			\operation{+ getResults(sliceSet : List\textless float[]\textgreater) : List\textless int[]\textgreater}
			\operation{//private methode's voor de werking van het algoritme}
		\end {class}
		
		\begin {class}[text width=8cm]{CrossCovarianceSyncStrategy}{1,-12}
			\inherit{SyncStrategy}
			\operation{+ findLatencies(sliceSet : List\textless float[]\textgreater) : List\textless Double\textgreater}
			\operation{//private methode's voor de werking van het algoritme}
		\end {class}
		
		\unidirectionalAssociation{[xshift=-2cm]CrossCovarianceSyncStrategy.north}{\hspace{-2.6cm}fingerprinter}{\hspace{-0.5cm}1}{FingerprintSyncStrategy.south}
	\end{package}
\end {tikzpicture}

	\end{center}
	\label{SyncStrategyUML}
\end{figure}

De synchronisatiealgoritmes bevinden zich in subklassen van \texttt{SyncStrategy}. Deze klasse bevat een abstracte methode \texttt{findLatencies}. Deze methode ontvangt als parameter een lijst van \texttt{float} arrays waarbij elke array de samples bevat van één slice bevat. Na het uitvoeren geeft de methode een lijst met \texttt{LatencyResults} terug.

\texttt{SyncStrategy} bevat ook enkele statische methodes waarmee de verschillende mogelijke strategieën gemakkelijk kunnen worden aangemaakt. De parameters die aan de constructoren moeten worden meegegeven worden uit het configuratiebestand gehaald. 

De klasse \texttt{CrossCovarianceSyncStrategy} bevat een instantie van \texttt{FingerprintSyncStrategy} om de geleverde resultaten te kunnen verfijnen. Beide klassen erven over van \texttt{SyncStrategy}.

De klasse \texttt{CrossCovarianceSyncStrategy} bevat informatie over de bepaalde latency.

Figuur \ref{SyncStrategyUML} toont het UML diagram van de besproken klassen.

\section*{Bepalen van de latency}

\begin{figure}[h!]
	\captionsetup{width=0.7\textwidth}
	\caption[UML diagram van \texttt{RealtimeSignalSync} + afhankelijkheden]{UML diagram van \texttt{RealtimeSignalSync} en alle klassen waarvan ze afhankelijk is.}
	\begin{center}
		\advance\parskip0.3cm
		\begin {tikzpicture}% [ show background grid ]
	\begin{package}[StreamPackage]{be.signalsync.sync}

		\begin{interface}[text width=12cm, template parameter=<T>]{SliceListener}{0,0}
		
		\end{interface}

		\begin{class}[text width=14cm]{RealtimeSignalSync}{0,-3}
			\implement{Stream}
			\operation{+ RealtimeSignalSync(streamSet : StreamSet)}
			\operation{+ onSliceEvent(event : SliceEvent\textless Map\textless StreamGroup, float[]\textgreater\textgreater) : void}
			\operation{+ addEventListener(listener : SyncEventListener) : void}
			\operation{+ removeEventListener(listener : SyncEventListener) : void}
			\operation{- emitSyncEvent(data : Map\textless StreamGroup, Double\textgreater) : void}
		\end{class}
		
		\begin{interface}[text width=10.5cm]{SyncEventListener}{0,-14}
			\operation[0]{onSyncEvent(data : Map\textless StreamGroup, Double\textgreater) : void}
		\end{interface}
		
		\begin{abstractclass}[text width=5cm]{SyncStrategy}{-4,-10}
		\end{abstractclass}
		
		\begin{class}[text width=5cm]{StreamSetSlicer}{4,-9.5}
		\end{class}
		
		\begin{class}[text width=5cm]{StreamSet}{3,-12}
		\end{class}
				
		
		\coordinate [label={[align=center]\guillemotleft \textbf{bind}\guillemotright \\[-0.7em] T $\rightarrow$ Map\textless StreamGroup,float\lbrack \rbrack\textgreater}] (D) at (3.2,-2.6);
		
		\unidirectionalAssociation{RealtimeSignalSync}{\hspace{-1.8cm}listeners}{\hspace{-0.9cm}0..*}{SyncEventListener}
		
		\unidirectionalAssociation{[xshift=-4cm]RealtimeSignalSync.south}{\hspace{-1.7cm}strategy}{\hspace{-0.5cm}1}{SyncStrategy.north}
		
		\unidirectionalAssociation{[xshift=4cm]RealtimeSignalSync.south}{\hspace{1.1cm}slicer}{\hspace{0.7cm}1}{StreamSetSlicer.north}
		
		\unidirectionalAssociation{[xshift=0.7cm]RealtimeSignalSync.south}{\hspace{2cm}streamSet}{\hspace{0.5cm}1}{[xshift=-2.3cm]StreamSet.north}
	\end{package}
\end {tikzpicture}

	\end{center}
	\label{latencyUML}
\end{figure}

De klasse \texttt{RealtimeSignalSync} is verantwoordelijk voor het aanroepen van de klassen uit verschillende packages waardoor de uiteindelijke synchronisatie kan plaatsvinden.

Een \texttt{RealtimeSignalSync} object wordt aangemaakt door aan de constructor een \texttt{StreamSet} object mee te geven dat de te synchroniseren streams bevat. Vervolgens wordt er met behulp van de methode \texttt{createSlicer} een \texttt{StreamSetSlicer} object aangemaakt. Aangezien \texttt{SliceListener} geïmplementeerd wordt kan \texttt{RealtimeSignalSync} zich registreren als geïnteresseerd object.

Wanneer er voor elke \texttt{StreamGroup} een slice beschikbaar is wordt de methode \texttt{onSliceEvent} opgeroepen. In deze methode wordt het \texttt{SyncStrategy} object gebruikt om de latencies te bepalen. Optioneel kunnen deze resultaten ook nog gefilterd worden. Ten slotte worden de resultaten naar alle geregistreerde \texttt{SyncEventListeners} verstuurd via de private methode \texttt{emitSyncEvent}. Geïnteresseerden kunnen zich registreren via de \texttt{addEventListener} methode. Deze werkwijze leunt net zoals het \textit{slice event} mechanisme aan bij het \textit{observer} ontwerppatroon.

\subsubsection{Het starten van de synchronisatie}

Het starten van de synchronisatie gebeurt noch in deze klasse noch in een slicer- of \mbox{strategyklasse}. Zowel het slicen als het uitvoeren van de algoritmes is een deterministisch proces en afhankelijk van de samples die door een \texttt{Stream} object naar de \texttt{StreamProcessors} verstuurd worden. Bij een \texttt{AudioDispatcherStream} zal de synchronisatie starten wanneer de \texttt{start} methode opgeroepen wordt. Bij een \texttt{MSPStream} zal de synchronisatie pas starten wanneer de streams in Max/MSP zelf geactiveerd worden.

\section*{Filteren van de resultaten}
\label{filterontwerp}

\begin{figure}[h!]
	\captionsetup{width=0.7\textwidth}
	\caption{UML diagram van de verschillende datafilters.}
	\begin{center}
		\advance\parskip0.3cm
		\begin {tikzpicture}[scale=0.7]
\tikzstyle{every node}=[font=\scriptsize]
\begin{package}[StreamPackage]{be.signalsync.sync}
	\begin{interface}[text width=7cm]{DataFilter}{0,0}
		\operation[0]{filter(rawValue : double) : double}
	\end{interface}
	
	\begin{class}[text width=5cm]{NoFilter}{4,-3}
		\implement{DataFilter}
		\operation{\# filter(double rawValue) : double}
	\end{class}
	
	\begin{abstractclass}[text width=6cm]{BufferedFilter}{0,-5}
		\implement{DataFilter}
		\attribute{\# buffer : Queue\textless Double\textgreater}
		\attribute{\# bufferSize : int}
		\attribute{\# currentValue : double}
		\attribute{- initialized : boolean}
		\operation{+ filter(rawValue : double) : double}
		\operation{- initializeBuffer(rawValue : double) : void}
		\operation[0]{\# calculateNext(double rawValue) : double}
	\end{abstractclass}
	
	\begin{class}[text width=6cm]{MovingAverageFilter}{-2,-13}
		\inherit{BufferedFilter}
		\operation{\# calculateNext(double rawValue) : double}
	\end{class}
	
	\begin{class}[text width=6cm]{MovingMedianFilter}{2,-15}
		\inherit{BufferedFilter}
		\operation{\# calculateNext(double rawValue) : double}
	\end{class}		
	
\end{package}
\end {tikzpicture}

	\end{center}
	\label{filterUML}
\end{figure}

Zoals in sectie \ref{filtering} is besproken kunnen de latencies gefilterd worden. De meest elementaire filter wordt beschreven door de interface \texttt{DataFilter} waarin de methode \texttt{filter} wordt beschreven. Deze methode ontvangt een \texttt{double} als parameter en geeft een (gefilterde) \texttt{double} terug. De meest eenvoudige implementatie van deze klasse is de \texttt{NoFilter}. Deze filter geeft rechtstreeks de meegegeven waarde terug zonder iets te wijzigen.

\subsubsection{Gebufferde filters}

\texttt{DataFilter} wordt ook geïmplementeerd door de abstracte klasse \texttt{BufferedFilter}. Dit is een filter die de laatste $ n $ waarden in een buffer bijhoudt. De waarde van $ n $ wordt bepaalt in het configuratiebestand. 

Deze klasse is abstract aangezien de manier waarop de uiteindelijk gefilterde waarde \mbox{berekent} wordt nog niet bepaald is. Hiervoor is de abstracte methode \texttt{calculateNext} voorzien. De klasse \texttt{MovingAverageFilter} doet dit door het gemiddelde van de waarden uit de buffer te nemen. Bij \texttt{MovingMedianFilter} gebeurt dit door de mediaan te berekenen.

Figuur \ref{filterUML} toont een UML diagram van deze verschillende klassen en interfaces.