\chapter{Testresultaten: gewijzigde latency}
\label{appendix-c}

\section*{Dataset}

Alle testen worden uitgevoerd ten opzichte van een referentie audiofragment. Er zijn 12 varianten voorzien met een elk een andere latency. Dit is de verzameling van verschillende latencies:
\begin{center}
	\{$20ms$, $-20ms$, $80ms$, $-80ms$, $90ms$, $-90ms$, $300ms$, $-300ms$, \\$2000ms$, $-2000ms$, $6000ms$, $-6000ms$ \}
\end{center}

Van elk audiofragment (referentie en elke variant) worden (ongeveer) 55 slices aangemaakt (lengte: 10s, overlap: 5s). Elke slice wordt gematcht met de corresponderende slice van het referentie audiofragment.

\section*{Test 1}

\subsection*{Instellingen}

\begin{tabular}{ l  l}
	\hline
	\textbf{Instelling} & \textbf{Waarde} \\
	\hline
	Algoritme & Accoustic fingerprinting \\
	\texttt{NFFT\_EVENT\_POINT\_MIN\_DISTANCE} & 100 \\
	\texttt{NFFT\_MAX\_FINGERPRINTS\_PER\_EVENT\_POINT} & 10 \\
	\texttt{MIN\_ALIGNED\_MATCHES} & 2 \\
	Nauwkeurigheid & $32ms$ \\
	Opmerking & Standaard parameters \\
	\\
\end{tabular}\\

\subsection*{Resultaten}

\begin{tabular}{ l  l}
	\hline
	\textbf{Resultaat} & \textbf{Waarde} \\
	\hline
	Totaal aantal testen & 324 \\
	Totaal aantal geslaagd & 324 \\
	Totaal aantal foutief & 0 \\
	Slaagpercentage & 100\% \\
	\\
\end{tabular}\\

\subsection*{Conclusie}

Zoals verwacht slagen alle testen.

\section*{Test 2}

\subsection*{Instellingen}

\begin{tabular}{ l  l}
	\hline
	\textbf{Instelling} & \textbf{Waarde} \\
	\hline
	Algoritme & Kruiscovariantie \\
	\texttt{NFFT\_EVENT\_POINT\_MIN\_DISTANCE} & 100 \\
	\texttt{NFFT\_MAX\_FINGERPRINTS\_PER\_EVENT\_POINT} & 10 \\
	\texttt{MIN\_ALIGNED\_MATCHES} & 2 \\
	\texttt{CROSS\_COVARIANCE\_NUMBER\_OF\_TESTS} & 1 \\
	\texttt{CROSS\_COVARIANCE\_THRESHOLD} & 1 \\
	Nauwkeurigheid & $0.1ms$ \\
	Opmerking & Eén kruiscovariantie test \\
	\\
\end{tabular}\\

\subsection*{Resultaten}

\begin{tabular}{ l  l}
	\hline
	\textbf{Resultaat} & \textbf{Waarde} \\
	\hline
	Totaal aantal testen & 324 \\
	Totaal geslaagd & 306 \\
	Totaal foutief & 18 \\
	Slaagpercentage & 94\% \\
	\\
\end{tabular}\\

\subsection*{Conclusie}

Sommige testen falen. Analyse toont aan dat dit enkel voorvalt wanneer minstens één kruiscovariantie buffer stilte bevat (samples zijn 0.0).

\section*{Test 3}

\subsection*{Instellingen}

\begin{tabular}{ l  l}
	\hline
	\textbf{Instelling} & \textbf{Waarde} \\
	\hline
	Algoritme & Kruiscovariantie \\
	\texttt{NFFT\_EVENT\_POINT\_MIN\_DISTANCE} & 100 \\
	\texttt{NFFT\_MAX\_FINGERPRINTS\_PER\_EVENT\_POINT} & 10 \\
	\texttt{MIN\_ALIGNED\_MATCHES} & 2 \\
	\texttt{CROSS\_COVARIANCE\_NUMBER\_OF\_TESTS} & 5 \\
	\texttt{CROSS\_COVARIANCE\_THRESHOLD} & 1 \\
	Nauwkeurigheid & $0.1ms$ \\
	Opmerking & Meerdere kruiscovariantie testen \\
	\\
\end{tabular}\\

\subsection*{Resultaten}

\begin{tabular}{ l  l}
	\hline
	\textbf{Resultaat} & \textbf{Waarde} \\
	\hline
	Totaal aantal testen & 324 \\
	Totaal geslaagd & 324 \\
	Totaal foutief & 0 \\
	Slaagpercentage & 100\% \\
	\\
\end{tabular}\\

\subsection*{Conclusie}

Na het verhogen van het aantal kruiscovariantie testen is het slaagpercentage 100\%.