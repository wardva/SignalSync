\chapter{Gebruikershandleiding}
\label{appendix-d}

Deze handleiding is bedoeld voor mensen (onderzoekers, musicologen,...) die gebruik willen maken van de Max/MSP module zonder enige kennis op het vlak van programmeren van software.

\section{Downloaden van de bestanden}



\section*{Het configuratiebestand}

Alle parameters van de algoritmen kunnen gewijzigd worden vanuit het configuratiebestand (\texttt{config.properties}). De betekenis van de parameters zijn al eerder besproken in deze scriptie en zullen daarom niet verder behandeld worden.

\section*{De Teensy microcontroller}
\label{read-teensy}

Om data of audio van een Teensy in te lezen in Max/MSP moet er een programma op de Teensy worden uitgevoerd.
De volgende GitHub map bevat twee Arduino sketches die op de Teensy kunnen worden uitgevoerd: \url{https://github.com/wardva/SignalSync/tree/master/Teensy\%20Arduino\%20sketches}. Op volgende webpagina wordt uitgelegd hoe Arduino sketches op een Teensy kunnen worden uitgevoerd: \url{https://www.pjrc.com/teensy/td_download.html}. 

De map \textbf{TeensyAnalogRead} bevat een sketch waarmee de analoge pinnen van een standaard Teensy met een frequentie van $8000Hz$ kunnen worden uitgelezen. In de map \textbf{MicToUSB} bevindt zich de code voor het inlezen van een Teensy uitgerust met een \textit{Audio Adaptor Board} aan een samplefrequentie van $11025Hz$. De geluidskwaliteit van een Teensy uitgerust met dergelijke apparatuur is opmerkelijk beter. Meer informatie hierover is te vinden op volgende pagina: \url{https://www.pjrc.com/store/teensy3_audio.html}.

\section*{Synchroniseren van streams in Max/MSP 7}

Vooraleer een module in Max/MSP kan worden ingeladen moet de locatie van de modules gespecificeerd worden. Het configuratiebestand bevindt zich in Windows onder volgende map (dit kan variëren afhankelijk van eigen instellingen): \texttt{C:\textbackslash Program Files\textbackslash Cycling '74\textbackslash Max 7\textbackslash resources\textbackslash packages\textbackslash max-mxj\textbackslash java-classes}. De naam van het bestand is \texttt{max.java.config.txt}. Het is sterk aangeraden om hiervan een back-up te nemen want het bestand is erg foutgevoelig. Voeg in het bestand de volgende lijn toe: \texttt{max.dynamic.jar.dir <<pad>>}. Hierbij moet \texttt{<<pad>>} worden vervangen door het pad van de map waarin zich de JAR bestanden en het configuratiebestand bevinden. Backslashes moeten met een backslash geëscapet worden. Hierna zou het mogelijk moeten zijn om de eigen Max/MSP modules te gebruiken.

\subsection*{De TeensyReader module}
Hoe deze module wordt aangemaakt is in sectie \ref{teensy-reader} al uitgebreid besproken. Bij het inladen van deze module is het aangeraden (maar zeker niet verplicht) om de samplefrequentie van Max/MSP in te stellen op een veelvoud van de samplefrequentie van de Teensy. Dit vereenvoudigt het resampleproces. Soms ontstaat er namelijk drift wanneer de frequenties niet eenvoudig kunnen worden omgezet.


\subsection*{De Sync module}

Deze module is in sectie \ref{sync-module} al in detail besproken. Net zoals bij de vorige module moeten de streams worden geresamplet. Het is daarom aangeraden om er voor te zorgen dat de samplefrequentie van Max/MSP een veelvoud is van de samplefrequentie gebruikt in de softwarebibliotheek (\texttt{SAMPLE\_RATE} in het configuratiebestand).