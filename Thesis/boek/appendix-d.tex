\chapter{Gebruikershandleiding}
\label{appendix-d}

Deze handleiding is bedoeld voor mensen (onderzoekers, musicologen,...) die gebruik willen maken van de Max/MSP module zonder enige kennis op het vlak softwareprogrammatie.

\subsection*{Installatie}

Eerst moeten de JAR bestanden en het configuratiebestand gedownload worden van het volgende GitHub repository: \url{https://github.com/wardva/SignalSync/tree/master/Dist}. De bestanden zijn ook rechtstreeks (gezipt) te downloaden via volgende link: \url{https://github.com/wardva/SignalSync/raw/master/Dist/Dist.zip}. Een gratis tool voor het uitpakken van het archief is \href{http://www.7-zip.org/}{7Zip}.

Na het binnenhalen (en eventueel uitpakken) van de bestanden dienen ze te worden verplaatst naar de \emph{lib} map in het installatiedirectory van Max/MSP. Dit is de standaardlocatie van deze map onder Windows: \texttt{C:\textbackslash Program Files\textbackslash Cycling '74\textbackslash Max 7\textbackslash resources\\\textbackslash packages\textbackslash max-mxj\textbackslash java-classes\textbackslash lib}. De plaats kan variëren afhankelijk van eigen gebruikersinstellingen.

Na het herstarten van Max/MSP kunnen de modules gebruikt worden (zie verder).

\subsection*{Het configuratiebestand}

Indien het configuratiebestand (\texttt{config.properties}) zich in dezelfde map bevindt als de JAR bestanden kunnen de parameters van de algoritmes gewijzigd worden. Na elke wijziging is het noodzakelijk om Max/MSP te herstarten. De betekenis van de parameters zijn al eerder besproken in deze scriptie en zullen daarom niet verder behandeld worden.

\subsection*{Aanmaken van een module}

Een in Java geschreven module wordt in Max/MSP aangemaakt met behulp van de \texttt{mxj} module. Een module die aan digitale signaalverwerking doet wordt aangemaakt met behulp van de \texttt{mxj\textapprox} module. De \texttt{mxj(\textapprox)} module moet worden opgeroepen met als parameter de volledige \emph{klassenaam} van de Java module. Na deze naam volgen de module-specifieke parameters (deze zijn eerder in deze scriptie al besproken).

De namen van de modules die resulteren uit dit onderzoek: \texttt{be.signalsync.msp.Sync} en \texttt{be.signalsync.msp.TeensyReader}. Beide modules doen aan digitale signaalverwerking en moeten dus worden aangemaakt met \texttt{mxj\textapprox}.


\subsection*{De TeensyReader module}
Deze module is in sectie \ref{teensy-reader} al uitgebreid besproken. Bij het inladen van deze module is het aangeraden (maar zeker niet verplicht) om de samplefrequentie van Max/MSP in te stellen op een veelvoud van de samplefrequentie van de Teensy. Dit vereenvoudigt het resampleproces. Soms ontstaat er drift wanneer de frequenties niet eenvoudig kunnen worden omgezet.

\subsection*{De Sync module}

Deze module is in sectie \ref{sync-module} al in detail besproken. Net zoals bij de vorige module moeten de streams worden geresamplet. Het is daarom aangeraden om er voor te zorgen dat de samplefrequentie van Max/MSP een veelvoud is van de samplefrequentie gebruikt in de softwarebibliotheek (\texttt{SAMPLE\_RATE} in het configuratiebestand).

\section*{De Teensy microcontroller}
\label{read-teensy}

Om data of audio van een Teensy in te lezen in Max/MSP moet er een programma op de Teensy worden uitgevoerd.
De volgende GitHub map bevat twee Arduino sketches die op de Teensy kunnen worden uitgevoerd: \url{https://github.com/wardva/SignalSync/tree/master/Teensy\%20Arduino\%20sketches}. Op volgende webpagina wordt uitgelegd hoe Arduino sketches op een Teensy kunnen worden uitgevoerd: \url{https://www.pjrc.com/teensy/td_download.html}. 

De map \textbf{TeensyAnalogRead} bevat een sketch waarmee de analoge pinnen van een standaard Teensy met een frequentie van $8000Hz$ kunnen worden uitgelezen. In de map \textbf{MicToUSB} bevindt zich de code voor het inlezen van een Teensy uitgerust met een \textit{Audio Adaptor Board} aan een samplefrequentie van $11025Hz$. De geluidskwaliteit van een Teensy uitgerust met dergelijke apparatuur is opmerkelijk beter. Meer informatie hierover is te vinden op volgende pagina: \url{https://www.pjrc.com/store/teensy3_audio.html}.