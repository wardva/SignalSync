%  Overzichtsbladzijde met samenvatting

\newpage

{
\setlength{\baselineskip}{14pt}
\setlength{\parindent}{0pt}
\setlength{\parskip}{8pt}

\begin{center}

\noindent \textbf{\huge
Realtime signaal synchronisatie\\[8pt]
met acoustic fingerprinting
}

door 

Ward Van Assche

Masterproef ingediend tot het behalen van de academische graad van\\
Master of Science in de industriële wetenschappen: informatica

Academiejaar 2015--2016

Promotoren: Dr.~Marleen Denert, Joren Six\\
Scriptiebegeleider: Prof.~Helga Naessens

Faculteit Ingenieurswetenschappen en Architectuur\\
Universiteit Gent

Vakgroep Informatietechnologie\\
Voorzitter: Prof.~Dr.~Ir.~Dani\"{e}l De Zutter


\end{center}

\section*{Samenvatting}

% TODO: samenvatting schrijven

De meeste experimenten die aan het IPEM worden uitgevoerd maken gebruik van verschillende soorten sensoren (accelerometers, druksensoren,...). Een veelvoorkomend probleem is de de synchronisatie van de data van elke sensor. Bij het huidige synchronisatiesysteem wordt elke sensor verbonden met een microfoon die het omgevingsgeluid opneemt. Met technieken zoals acoustic fingerprinting en het berekenen van de kruiscovariantie kan de latency tussen de audiosignalen zeer nauwkeurig bepaald worden. Met deze latency kan vervolgens de sensordata gesynchroniseerd worden. Het huidige systeem kan enkel als naverwerking uitgevoerd worden. Een realtime en meer gebruiksvriendelijk systeem is erg gewenst. In dit onderzoek is er nagegaan of dit mogelijk is. Om de huidige synchronisatiealgoritmen in realtime bruikbaar te maken waren aanpassingen en optimalisaties nodig. Dit onderzoek heeft ook geleid tot enkele Max/MSP modules. Met behulp van deze modules is het mogelijk om het volledige synchronisatieproces in realtime uit te voeren in zonder het schrijven van één lijn code.


\section*{Trefwoorden}

% TODO: trefwoorden

synchronisatie, sensoren, audio, geluid, realtime, signalen, streams, acoustic fingerprinting, kruiscovariantie, digitale signaalverwerking

}
\newpage % strikt noodzakelijk om een header op deze blz. te vermijden
