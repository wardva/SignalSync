\chapter{Testresultaten: toevoegen van een sinusgolf}
\label{appendix-d}

\section*{Dataset}

Bij dit experiment wordt de dataset van bijlage \ref{appendix-b} uitgebreid. Van elke latency uit de eerder besproken dataset worden twee varianten voorzien. Aan elke variant wordt één van deze sinusgolven toegevoegd waardoor de golfvorm er helemaal anders gaat uitzien:

\begin{center}
	\{ $50Hz$, $100Hz$ \}
\end{center}

Van deze aangepaste audiofragmenten worden op voorhand opnieuw slices aangemaakt. 

\section*{Test 1}

\subsection*{Instellingen}

\begin{tabular}{ l  l}
	\hline
	\textbf{Instelling} & \textbf{Waarde} \\
	\hline
	Algoritme & Accoustic fingerprinting \\
	\texttt{NFFT\_EVENT\_POINT\_MIN\_DISTANCE} & 100 \\
	\texttt{NFFT\_MAX\_FINGERPRINTS\_PER\_EVENT\_POINT} & 10 \\
	\texttt{MIN\_ALIGNED\_MATCHES} & 2 \\
	Nauwkeurigheid & $32ms$ \\
	Opmerking & Standaard parameters \\
\end{tabular}\\

\subsection*{Resultaten}

\begin{tabular}{ l  l}
	\hline
	\textbf{Resultaat} & \textbf{Waarde} \\
	\hline
	Totaal aantal testen & 648 \\
	Totaal aantal geslaagd & 648 \\
	Totaal aantal foutief & 0 \\
	Slaagpercentage & 100\% \\
\end{tabular}\\

\subsection*{Conclusie}

Het accoustic fingerprinting algoritme heeft geen problemen met het toevoegen van de sinusgolven aan één van de audiofragmenten.

\section*{Test 2}

\subsection*{Instellingen}

\begin{tabular}{ l  l}
	\hline
	\textbf{Instelling} & \textbf{Waarde} \\
	\hline
	Algoritme & Kruiscovariantie \\
	\texttt{NFFT\_EVENT\_POINT\_MIN\_DISTANCE} & 100 \\
	\texttt{NFFT\_MAX\_FINGERPRINTS\_PER\_EVENT\_POINT} & 10 \\
	\texttt{MIN\_ALIGNED\_MATCHES} & 2 \\
	\texttt{CROSS\_COVARIANCE\_NUMBER\_OF\_TESTS} & 1 \\
	\texttt{CROSS\_COVARIANCE\_THRESHOLD} & 1 \\
	Nauwkeurigheid & $0.1ms$ \\
	Opmerking & Eén kruiscovariantie test \\
\end{tabular}\\

\subsection*{Resultaten}

\begin{tabular}{ l  l}
	\hline
	\textbf{Resultaat} & \textbf{Waarde} \\
	\hline
	Totaal aantal testen & 648 \\
	Totaal aantal geslaagd & 465 \\
	Totaal aantal foutief & 183 \\
	Slaagpercentage & 72\% \\
\end{tabular}\\

\subsection*{Conclusie}

Het kruiscovariantie ondervindt, bij éénmalige uitvoering, serieuze problemen bij het bepalen van de latency wanneer er een sinusgolf is toegevoegd.

\section*{Test 3}

\subsection*{Instellingen}

\begin{tabular}{ l  l}
	\hline
	\textbf{Instelling} & \textbf{Waarde} \\
	\hline
	Algoritme & Kruiscovariantie \\
	\texttt{NFFT\_EVENT\_POINT\_MIN\_DISTANCE} & 100 \\
	\texttt{NFFT\_MAX\_FINGERPRINTS\_PER\_EVENT\_POINT} & 10 \\
	\texttt{MIN\_ALIGNED\_MATCHES} & 2 \\
	\texttt{CROSS\_COVARIANCE\_NUMBER\_OF\_TESTS} & 5 \\
	\texttt{CROSS\_COVARIANCE\_THRESHOLD} & 1 \\
	Nauwkeurigheid & $0.1ms$ \\
	Opmerking & Meerdere kruiscovariantie testen
\end{tabular}\\

\subsection*{Resultaten}

\begin{tabular}{ l  l}
	\hline
	\textbf{Resultaat} & \textbf{Waarde} \\
	\hline
	Totaal aantal testen & 648 \\
	Totaal aantal geslaagd & 642 \\
	Totaal aantal foutief & 6 \\
	Slaagpercentage & 99\% \\
\end{tabular}\\

\subsection*{Conclusie}

Het resultaat dat verkregen wordt wanneer het algoritme vijf maal wordt uitgevoerd staat in groot contrast met vorig resultaat. Een slaagpercentage van 99\% is zeker aanvaardbaar voor deze toepassing.

\section*{Test 4}

\subsection*{Instellingen}

\begin{tabular}{ l  l}
	\hline
	\textbf{Instelling} & \textbf{Waarde} \\
	\hline
	Algoritme & Kruiscovariantie \\
	\texttt{NFFT\_EVENT\_POINT\_MIN\_DISTANCE} & 100 \\
	\texttt{NFFT\_MAX\_FINGERPRINTS\_PER\_EVENT\_POINT} & 10 \\
	\texttt{MIN\_ALIGNED\_MATCHES} & 2 \\
	\texttt{CROSS\_COVARIANCE\_NUMBER\_OF\_TESTS} & 20 \\
	\texttt{CROSS\_COVARIANCE\_THRESHOLD} & 1 \\
	Nauwkeurigheid & $0.1ms$ \\
	Opmerking & Een groot aantal kruiscovariantie testen
\end{tabular}\\

\subsection*{Resultaten}

\begin{tabular}{ l  l}
	\hline
	\textbf{Resultaat} & \textbf{Waarde} \\
	\hline
	Totaal aantal testen & 648 \\
	Totaal aantal geslaagd & 648 \\
	Totaal aantal foutief & 0 \\
	Slaagpercentage & 100\% \\
\end{tabular}\\

\subsection*{Conclusie}

Een optimaal resultaat wordt bereikt door de test 20 maal uit te voeren.