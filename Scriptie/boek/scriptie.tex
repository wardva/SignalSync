\documentclass[12pt,a4paper,oneside]{book}
\usepackage{a4wide}                     % Iets meer tekst op een bladzijde
\usepackage[dutch,english]{babel}       % Voor nederlandstalige hyphenatie (woordsplitsing)
\usepackage{amsmath}                    % Uitgebreide wiskundige mogelijkheden
\usepackage{amssymb}                    % Voor speciale symbolen zoals de verzameling Z, R...
\usepackage{url}                        % Om url's te verwerken
\usepackage{graphicx}                   % Om figuren te kunnen verwerken
\usepackage[small,bf,hang]{caption2}    % Om de captions wat te verbeteren
\usepackage{xspace}                     % Magische spaties na een commando
\usepackage[utf8]{inputenc}           	% Om niet ascii karakters rechtstreeks te kunnen typen
\usepackage{float}                      % Om nieuwe float environments aan te maken. Ook optie H!
\usepackage{flafter}                    % Opdat floats niet zouden voorsteken
\usepackage{listings}                   % Voor het weergeven van letterlijke text en codelistings
\usepackage{marvosym}                   % Om het euro symbool te krijgen
\usepackage{textcomp}                   % Voor onder andere graden celsius
\usepackage{fancyhdr}                   % Voor fancy headers en footers.
\usepackage{graphics}					% Om figuren te verwerken.
\usepackage[nottoc]{tocbibind} 			% Bibliografie in ToC; zie tocbibind.dvi
%\usepackage{pstricks}
\usepackage{longtable}
\usepackage{pdfpages}  					% pdf pagina's importeren
\usepackage[numbers]{natbib}			% Extra citeer mogelijkheden
\usepackage{parskip}					% Geen indentatie bij begin paragraaf
\usepackage[para]{footmisc}				%

\newcommand{\npar}{\par \vspace{2.3ex plus 0.3ex minus 0.3ex} \noindent}	% Om witruimte te krijgen tussen paragrafen
\graphicspath{{figuren/}}               % De plaats waar latex zijn figuren gaat halen.
\usepackage[bf]{caption2}	% Mooiere captions
\usepackage[a4paper,plainpages=false]{hyperref}    % Om hyperlinks te hebben in het pdfdocument.
\newcommand{\command}[1]{\lstinline[basicstyle=\tt]{#1}\xspace} %Voor commando's
\hyphenation{}

\renewcommand{\chaptermark}[1]{\markright{\MakeUppercase{#1}}}
\renewcommand{\sectionmark}[1]{\markright{\thesection~#1}}

\newcommand{\headerfmt}[1]{\textsl{\textsf{#1}}}
\newcommand{\headerfmtpage}[1]{\textsf{#1}}

\fancyhf{}
\fancyhead[LE,RO]{\headerfmtpage{\thepage}}
\fancyhead[LO]{\headerfmt{\rightmark}}
\fancyhead[RE]{\headerfmt{\leftmark}}
\renewcommand{\headrulewidth}{0.5pt}
\renewcommand{\footrulewidth}{0pt}

\fancypagestyle{plain}{ % eerste bladzijde van een hoofdstuk
	\fancyhf{}
	\fancyhead[LE,RO]{\headerfmtpage{\thepage}}
	\fancyhead[LO]{\headerfmt{\rightmark}}
	\fancyhead[RE]{\headerfmt{\leftmark}}
	\renewcommand{\headrulewidth}{0.5pt}
	\renewcommand{\footrulewidth}{0pt}
}

\renewcommand{\lstlistoflistings}{\begingroup
	\tocfile{\lstlistlistingname}{lol}
	\endgroup}


% anderhalve interlinie (opm: titelblad gaat uit van 1.5)
\renewcommand{\baselinestretch}{1.5}


\hypersetup {
	pdfauthor = {Ward Van Assche},
	pdftitle = {Realtime signaal synchronisatie	met accoustic fingerprinting},
	pdfsubject = {Masterproef ingediend tot het behalen van de academische graad van Master of Science in de industriële wetenschappen: informatica, juni 2016},
	colorlinks = False,
%	pdfborder = {0 0 0}
}

\renewcommand\lstlistlistingname{Lijst van codefragmenten}
\renewcommand\lstlistingname{Codefragment}
\addto{\captionsdutch}{\renewcommand{\bibname}{Referentielijst}}

\begin{document}
\selectlanguage{dutch}

% titelblad (voor kaft)
%  Titelblad

% Opmerking: gaat uit van een \baselinestretch waarde van 1.5 (die moet
% ingesteld worden voor het begin van de document environment)

\begin{titlepage}

\setlength{\hoffset}{-1in}
\setlength{\voffset}{-1in}
\setlength{\topmargin}{1.5cm}
\setlength{\headheight}{0.5cm}
\setlength{\headsep}{1cm}
\setlength{\oddsidemargin}{3cm}
\setlength{\evensidemargin}{3cm}
\setlength{\footskip}{1.5cm}
\enlargethispage{1cm}
% \textwidth en \textheight hier aanpassen blijkt niet te werken

\fontsize{12pt}{14pt}
\selectfont

\begin{center}

\includegraphics[height=2cm]{ruglogo}

\vspace{0.5cm}

Faculteit Ingenieurswetenschappen en Architectuur\\
Vakgroep Informatietechnologie\\
Voorzitter: Prof.~Dr.~Ir.~Daniël De Zutter

\vspace{3cm}

\fontseries{bx}
\fontsize{17.28pt}{21pt}
\selectfont

Realtime signaal synchronisatie \\
met accoustic fingerprinting

\fontseries{m}
\fontsize{12pt}{14pt}
\selectfont

\vspace{.6cm}

door 

\vspace{.4cm}

Ward Van Assche

\vspace{2.5cm}

Promotoren: Dr.~Marleen Denert, Joren Six\\
Scriptiebegeleider: Prof.~Helga Naessens\\

\vspace{2cm}

Masterproef ingediend tot het behalen van de academische graad van\\
Master of Science in de industriële wetenschappen: informatica

\vspace{0.5cm}

Academiejaar 2015--2016

\end{center}
\end{titlepage}


% lege pagina (!!)

% titelblad (!!)

% geen paginanummering tot we aan de inhoudsopgave komen
\pagestyle{empty}

% voorwoord met dankwoord en toelating tot bruikleen (ondertekend)
%  Voorwoord (dankwoord) en toelating tot bruikleen

\newpage

\noindent \textbf{\huge Voorwoord}

\vspace{1.5cm}

\noindent
Lorem ipsum dolor sit amet, consectetur adipiscing elit. Praesent ac tristique risus. Morbi sit amet porta ex. Vivamus at blandit mi, eget finibus nisi. Cum sociis natoque penatibus et magnis dis parturient montes, nascetur ridiculus mus. Nulla vehicula efficitur rutrum. Quisque ac condimentum ligula, at tempus leo. Sed quis felis erat. Proin volutpat et odio ac finibus. Cras eu ultrices augue. Aliquam finibus lacus ut erat placerat, non ullamcorper arcu porttitor. Phasellus nisi metus, porttitor pretium diam ac, euismod ultricies nunc. Proin auctor pulvinar tellus eu egestas. Fusce non nisl commodo, luctus diam a, interdum ex. Pellentesque quis enim sit amet felis suscipit suscipit sed ac nunc. Integer porta, ipsum non placerat tincidunt, est quam eleifend urna, sed aliquam libero felis ac odio.\\

Proin non maximus est. Nullam eu magna et mauris faucibus dictum. Vivamus vitae commodo enim, dignissim consequat magna. Vivamus et elementum velit. Suspendisse justo lacus, euismod in nibh nec, dictum eleifend libero. Vivamus dapibus dignissim nibh, eu luctus nunc. Vivamus hendrerit massa in velit cursus pharetra. Aliquam consectetur dapibus mi at semper. \\

Nam rhoncus lectus risus. Vivamus dolor justo, viverra nec libero eget, porttitor rhoncus sapien. Proin commodo erat leo, eget dignissim massa consequat non. Sed pharetra eget libero nec faucibus. \\

\addvspace{2.5cm}

\noindent Ward Van Assche, juni 2016\newpage

\noindent \textbf{\huge Toelating tot bruikleen}

\vspace{1.5cm}

\noindent
``De auteur geeft de toelating deze scriptie voor consultatie beschikbaar
te stellen en delen van de scriptie te kopi\"eren voor persoonlijk
gebruik.\\
Elk ander gebruik valt onder de beperkingen van het auteursrecht,
in het bijzonder met betrekking tot de verplichting de bron uitdrukkelijk
te vermelden bij het aanhalen van resultaten uit deze scriptie.''

\addvspace{4cm}

\noindent Ward Van Assche, juni 2016


% overzicht
%  Overzichtsbladzijde met samenvatting

\newpage

{
\setlength{\baselineskip}{14pt}
\setlength{\parindent}{0pt}
\setlength{\parskip}{8pt}

\begin{center}

\noindent \textbf{\huge
Realtime signaal synchronisatie\\[8pt]
met acoustic fingerprinting
}

door 

Ward Van Assche

Masterproef ingediend tot het behalen van de academische graad van\\
Master of Science in de industriële wetenschappen: informatica

Academiejaar 2015--2016

Promotoren: Dr.~Marleen Denert, Joren Six\\
Scriptiebegeleider: Prof.~Helga Naessens

Faculteit Ingenieurswetenschappen en Architectuur\\
Universiteit Gent

Vakgroep Informatietechnologie\\
Voorzitter: Prof.~Dr.~Ir.~Dani\"{e}l De Zutter


\end{center}

\section*{Samenvatting}

% TODO: samenvatting schrijven

De meeste experimenten die aan het IPEM worden uitgevoerd maken gebruik van verschillende soorten sensoren (accelerometers, druksensoren,...). Een veelvoorkomend probleem is de de synchronisatie van de data van elke sensor. Bij het huidige synchronisatiesysteem wordt elke sensor verbonden met een microfoon die het omgevingsgeluid opneemt. Met technieken zoals acoustic fingerprinting en het berekenen van de kruiscovariantie kan de latency tussen de audiosignalen zeer nauwkeurig bepaald worden. Met deze latency kan vervolgens de sensordata gesynchroniseerd worden. Het huidige systeem kan enkel als naverwerking uitgevoerd worden. Een realtime en meer gebruiksvriendelijk systeem is erg gewenst. In dit onderzoek is er nagegaan of dit mogelijk is. Om de huidige synchronisatiealgoritmen in realtime bruikbaar te maken waren aanpassingen en optimalisaties nodig. Dit onderzoek heeft ook geleid tot enkele Max/MSP modules. Met behulp van deze modules is het mogelijk om het volledige synchronisatieproces in realtime uit te voeren in zonder het schrijven van één lijn code.


\section*{Trefwoorden}

% TODO: trefwoorden

synchronisatie, sensoren, audio, geluid, realtime, signalen, streams, acoustic fingerprinting, kruiscovariantie, digitale signaalverwerking

}
\newpage % strikt noodzakelijk om een header op deze blz. te vermijden


% abstract
\addcontentsline{toc}{chapter}{Extended abstract}
\includepdf[pages=-]{../abstract/abstract.pdf}
\pagestyle{fancy}

\frontmatter

% inhoudstafel
\tableofcontents

% afkortingen
\chapter{Gebruikte afkortingen}
\begin{flushleft}
	\renewcommand{\baselinestretch}{1.5}
	\small\normalsize
	\begin{longtable}{ll}
		IPEM				&  Instituut voor Psychoakoestiek en Elektronische Muziek \\
		DSP					&  Digital Signal Processing \\
		FFT					&  Fast Fourier Transform \\
		SFT					&  Short Time Fourier Transform \\
		ECG					&  Elektrocardiogram \\
		DTW					&  Dynamic timewarping \\
		USB					&  Universal Serial Bus \\
		ADC					&  Analog-to-digital converter \\
		PCM					&  Pulse-code modulation
		
	\end{longtable}
\end{flushleft}

% opmaak voor het eigenlijke boek; onderstaande lijnen
% weglaten als de eerste regel van een nieuwe alinea moet
% inspringen in plaats van extra tussenruimte
%\setlength{\parindent}{0pt}
%\setlength{\parskip}{0.5\baselineskip plus 0.5ex minus 0.2ex}
%\setlength{\parskip}{1ex plus 0.5ex minus 0.2ex}

% hoofdstukken
\mainmatter

% hier worden de hoofdstukken ingevoegd (\includes)
\chapter{Inleiding}

\section{Probleemschets}

Het probleem dat in deze masterproef zal worden onderzocht doet zich heel specifiek voor bij verschillende experimenten die aan het IPEM worden uitgevoerd. Het IPEM is de onderzoeksinstelling van het departement musicologie aan Universiteit Gent. De focus ligt vooral op onderzoek naar de interactie van muziek op fysieke aspecten van de mens zoals dansen, sporten en fysieke revalidatie.\cite{ipem2016}

Om de relatie tussen muziek en beweging te kunnen onderzoeken worden er allerhande sensoren gebruikt die bepaalde gebeurtenissen omzetten in analyseerbare data. Een mogelijk experiment (puur imaginair) kan bijvoorbeeld gebruik maken van een videocamera om een persoon te filmen en verschillende accelerometers om de bewegingen van de persoon te detecteren. Er wordt ook muziek afgespeeld zodat men met behulp van de videobeelden en de data van de accelerometers kan analyseren hoe de persoon reageert op de afgespeelde muziek.

Om het zojuist beschreven experiment verder te onderzoeken moeten er minstens drie datastromen worden geanalyseerd: de videobeelden, de data van de accelerometer(s), en de afgespeelde audio. Een probleem dat zich hierbij voordoet is de synchronisatie van deze verschillende datastromen. Om een goede analyse mogelijk te maken is het zeer gewenst dat men exact weet (tot op de milliseconde nauwkeurig) wanneer een bepaalde gebeurtenis in een datastroom zich heeft voorgedaan, zodat men deze gebeurtenis kan vergelijken met de gebeurtenissen in de andere datastromen.

Bij het IPEM maakt men gebruik van een systeem dat gebruikt maakt van audio opnames om de datastromen te kunnen synchroniserenn. Het principe werkt als volgt: men zorgt ervoor dat elke datastroom vergezeld wordt met een perfect gesynchroniseerde audiostroom, afkomstig van een opname van het omgevingsgeluid. In het voorgaande experiment is dit eenvoudig te verwezenlijken. Bij de videobeelden kan automatisch een audiospoor mee worden opgenomen. De accelerometer kan geplaatst worden op een microcontroller (bijvoorbeeld een Arduino), hierop kan een klein microfoontje geplaatst worden. Aangezien beide componenten zo dicht op de hardware geplaatst zijn is de latency tussen beide datastromen te verwaarlozen. 
%TODO bewijzen dat dit te verwaarlozen is.
De afgespeelde audio kan gebruikt worden als referentie, aangezien dit uiteraard al een perfecte weergave is van het omgevingsgeluid. Na het uitvoeren van het experiment beschikt men dus over drie datastromen, waarbij elke datastroom vergezeld is van een quasi perfect synchrone opname van het omgevingsgeluid (dat in de ruimte waar het experiment is uitgevoerd voor elke opname gelijk is). Het probleem van de synchronizatie van de verschillende datastromen kan bijgevolg gereduceerd worden tot het synchroniseren van de verschillende audiostromen.

Door de typisch eigenschappen van geluid is het helemaal niet zo moeilijk om verschillende audiostromen te synchroniseren. Bij het IPEM heeft men een bepaald systeem ontwikkeld dat dergelijke synchronisatie mogelijk maakt met behulp van \textit{accoustic fingerprinting}. Accoustic fingerprinting is vooral bekend van de enorm populaire smartphone app voor muziek identificatie: \textit{Shazam}. Hierbij wordt dit principe gebruikt om een kort stukje opgenomen audio te vergelijken met een gigantische database van akoestische fingerprints.\footnote{In deze scriptie wordt er verondersteld dat het principe achter accoustic fingerprinting gekend is bij de lezer. Een grondige theoretische bespreking van dit algoritme is te vinden in de paper van \citeauthor{Wang2003a}, één van de oprichters van Shazam ltd. De implementatie die bij het IPEM gebruikt wordt voor de synchronisatie van de audiostromen wordt verderop in deze scriptie besproken.} Na het uitvoeren van het fingerprinting algoritme is het mogelijk om een bijkomend algoritme uit te voeren namelijk: synchronisatie met \textit{kruiscovariantie}. Dit algoritme zorgt er voor dat de synchronisatie een veel nauwkeuriger resultaat oplevert.

Ondanks het feit dat er al een systeem bestaat om datastromen te synchroniseren zijn er in de praktijk toch nog heel wat beperkingen. De grootste beperking is dat het synchronisatieproces pas kan worden uitgevoerd wanneer het experiment is afgelopen, en dit volledig handmatig. De opgenomen audiobestanden moet worden verzameld op een computer, vervolgens kan met behulp van de audiobestanden de latency worden berekend tussen elke datastroom worden berekend. Vervolgens kunnen de datastromen worden gesynchroniseerd. Voor de musicologen die deze experimenten uitvoeren is deze werkwijze veel te omslachtig. Daarom is een eenvoudigere realtime systeem om de synchronisatie uit te voeren zeer gewenst.

Een ander probleem is iets vager en minder duidelijk te omschrijven. De resultaten van het kruiscovariantie algoritme bevatten soms afwijkingen die moeilijk te verklaren zijn. De precieze oorzaak hiervan, en hoe dit kan worden opgelost zal ook worden onderzocht. Ook is het kruiscovariantie algoritme in vergelijking met het accoustic fingerprinting algoritme véél gevoeliger voor storingen en ruis, veroorzaakt door slechte opnames. Aangezien de opnameapparatuur (zeker op microcontrollers) bij de uit te voeren experimenten vaak van slechte kwaliteit is, is het belangrijk om de algoritmes robuust genoeg te maken zodat ze hier niet over struikelen.

\section{Doel van het onderzoek}

Dit onderzoek wil drie zaken bereiken: 

\subsubsection{Optimalisatie van algoritmes}
Het testen en bug-vrij maken van de synchronisatiealgoritmes. Ook moeten ze eventueel worden aangepast zodat ze in een realtime implementatie gebruikt kunnen worden. Het beoogde doel is dat de algoritmes in staat moeten kunnen zijn om audio opgenomen met een basic microfoon op een microcontroller te synchroniseren met een nauwkeurigheid van minstens 1 milliseconde.

\subsubsection{Implementatie van een Java bibliotheek}
Het schrijven van een bibliotheek in Java die gebruik maakt van deze algoritmes om audiostromen te synchroniseren. Deze bibliotheek moet kunnen worden aangeroepen vanuit eender welke andere Java applicatie, en moet periodiek de huidige latency per audiostroom teruggeven.

\subsubsection{Implementatie van een MAX/MSP module}
De implementatie van een module in MAX/MSP\footnote{Cycling ‘74 Max/MSP is een softwarepakket en een visuele programmeertaal waarmee audio, video en multimedia kan worden verwerkt met behulp van onafhankelijke modules. Deze modules kunnen met elkaar worden om zo complexe zaken te bereiken. Buiten de standaard meegeleverde modules is het ook mogelijk om zelf modules te schrijven.\cite{maxmsp2016}} die realtime synchronisatie mogelijk moet maken via een interface die bruikbaar is voor musicologen/onderzoekers met een beperkte informatica achtergrond.

Deze module moet in staat zijn om verschillende datastromen als input binnen te krijgen, de synchronisatiebibliotheek aan te roepen, en de gesynchroniseerde datastromen als output terug te geven. Een andere Max module kan er dan voor zorgen dat deze data wordt weggeschreven naar een WAVE-bestand.

\section{Huidige implementatie}

Om het vervolg van deze scriptie goed te kunnen begrijpen is een introductie tot de bestaande toepassingen waarop wordt verder gebouwd, noodzakelijk. De twee belangrijkste bibliotheken zijn TarsosDSP en Panako. 

\subsection{TarsosDSP}
TarsosDSP\footnote{Deze bibliotheek is veel uitgebreider dan wat hier wordt geschreven. Deze paper geeft hierover meer informatie: \url{http://0110.be/files/attachments/415/aes53_tarsos_dsp.pdf}\cite{six2014tarsosdsp}} is een Java bibliotheek voor realtime audio analyse en verwerking ontwikkeld aan het IPEM door \citefullauthor{six2014tarsosdsp}. De bibliotheek bevat een groot aantal algoritmes voor audioverwerking en kan nog verder worden uitgebreid.

TarsosDSP werkt met een zeer eenvoudige \textit{processing pipeline}. We kunnen zelf een processing pipeline creëren door een instantie aan te maken van de klasse \texttt{AudioDispatcher}. Het aanmaken van een \texttt{AudioDispatcher} is het eenvoudigst met behulp van de klasse \texttt{AudioDispatcherFactory}. Deze Factory klasse voorziet in statische methodes om een \texttt{AudioDispatcher} aan te maken met als input een audiobestand, een array van \texttt{float} waarden, of de data van een microfoon. Aan deze pipeline kunnen vervolgens verschillende \texttt{AudioProcessors} worden toegevoegd. Een \texttt{AudioProcessors} moet verplicht de \texttt{process} en \texttt{processingFinished} methodes implementeren. De \texttt{process} methode heeft als parameter een object van de klasse \texttt{AudioEvent}. Dit object bevat een audio blok, voorgesteld als \texttt{float} array met waarden tussen -1.0 en 1.0. De grootte van dit blokje audio, en de mate van overlapping tussen de opeenvolgende blokjes audio is instelbaar. Verder bevat het AudioEvent object ook nog andere metadata zoals onder meer een \textit{timestamp}.

Afhankelijk van de implementatie van de \texttt{process} methode kan de audiostroom op een bepaalde manier verwerkt, geanalyseerd of gewijzigd worden.

\subsection{Panako}

Panako\footnote{Ook deze bibliotheek wordt meer uitvoerig besproken in een paper geschreven door de auteurs: \url{http://0110.be/files/attachments/415/ismir_2014_panako_fingerprinter.pdf}\cite{six2014panako}} is ook een Java bibliotheek, samen met enkele applicaties die deze bibliotheek aanroepen. Panako is ook ontwikkeld aan het IPEM door \citefullauthor{six2014panako} en maakt gebruik van TarsosDSP als onderliggende technologie voor de verwerking van de audio.

Panako bevat een open-source implementatie van het accoustic fingerprinting algoritme beschreven in de paper van \citefullauthor{Wang2003a}. Dit algoritme is verder uitgebreid zodat audio waarbij de toonhoogte verhoogd of verlaagd is, of de audio sneller of trager is afgespeeld toch gedetecteerd kan worden.

De bibliotheek bevat verschillende applicaties die gebruik maken van dit algoritme. Zo is het mogelijk om de fingerprints van een geluidsfragment te bekijken, matches tussen verschillende geluidsfragmenten te visualiseren, en grafisch te experimenteren met de verschillende parameters.

Er is ook een applicatie beschikbaar om verschillende geluidsfragmenten te synchroniseren. Deze applicatie maakt buiten van het accoustic fingerprinting algoritme ook nog gebruik van het kruiscovariantie algoritme, dit is ook geïmplementeerd in Panako. Op welk principe dat dit algoritme gebaseerd is, en hoe dit geïmplementeerd is wordt verder in deze scriptie besproken. Wanneer de latency tussen de verschillende fragmenten is gedetecteerd genereert de applicatie een shell script dat met behulp van FFmpeg\footnote{Dit is een command-line applicatie voor het opnemen, verwerken en streamen van audio en video. Meer informatie: \url{https://ffmpeg.org/}} stukjes van de geluidsbestanden wegknipt of stilte toevoegt. Het resultaat is dat na het uitvoeren van het script de geluidsbestanden gesynchroniseerd zijn.


\chapter{Applicatie-ontwerp}

% Volledige ontwerp van de applicatie uitleggen, gebruik van dispatchers,
% multithreading, hoe werkt het realtime
\chapter{Accoustic fingerprinting}

\section{Optimalisaties}

\section{Parameterisatie}

\section{Uitgevoerde testen}
\chapter{Kruiscovariantie}

\section{Optimalisaties}

%Bugfixes: onderschatten en overschatten van fingerprint maakt verschil
%Verschillende keren per slice uitvoeren, threshold
%Bandpass filtering die toch niet geïmplementeerd is.

\section{Parameterisatie}

\section{Uitgevoerde testen}

%Experiment met inmixen van een sinusgolf, uitvoeren van unittest

\chapter{Implementatie in MAX/MSP}
\include{h6-conclusie}

% appendices
\appendix

% Bibliografie toevoegen
\bibliographystyle{plainnat}
\bibliography{../bronnen/bronnen}

\backmatter

% eventueel: lijst van uren en tabellen
\listoffigures
\addcontentsline{toc}{chapter}{Lijst van figuren}

\listoftables
\addcontentsline{toc}{chapter}{Lijst van tabellen}

\lstlistoflistings
\addcontentsline{toc}{chapter}{Lijst van codefragmenten}

% lege pagina (!!)

% kaft

\end{document}
