\chapter{Implementatie}

\section{Technologieën en software}

\subsection{Java 7}

Er zijn verschillende redenen waarom er gekozen is om de synchronisatie bibliotheek in Java te implementeren. De belangrijkste reden is dat de bestaande audio bibliotheken (Panako en TarsosDSP) van het IPEM ook ontwikkeld zijn in Java. Deze kunnen enkel aangeroepen worden via andere Java applicaties.

Het is ook de bedoeling is om de gebruikersinterface te ontwikkelen met behulp van Max/MSP modules. Het ontwikkelen van dergelijke modules is mogelijk in twee programmeertalen: C en Java. De eenvoud en hoge graad van abstractie gaf de doorslag om hierbij Java te gebruiken. De voordelen die C biedt op vlak van snelheid wegen (in deze toepassing) hier niet tegenop.

Hoewel de laatste versie van Max/MSP (7) het toelaat om Java 8 te gebruiken wordt deze versie in dit project om compatibiliteitsredenen vermeden. 



\subsection{TarsosDSP}

TarsosDSP is een Java bibliotheek voor realtime audio analyse en verwerking ontwikkeld aan het IPEM. De bibliotheek bevat een groot aantal algoritmes voor audioverwerking en kan nog verder worden uitgebreid. Deze bibliotheek wordt beschreven in artikel \cite{six2014tarsosdsp}. 

TarsosDSP is voornamelijk gebouwd rond het concept \textit{processing pipeline}. Dit is een abstractie van een audiostream die via programmacode verwerkt kan worden.  Een processing pipeline wordt voorgesteld als instantie van de klasse \texttt{AudioDispatcher}. Het aanmaken gebeurt met behulp van de klasse \texttt{AudioDispatcherFactory}. Deze bevat statische methodes om een \texttt{AudioDispatcher} aan te maken van een audiobestand, een array van floating-point getallen of een microfoon. Een processing pipeline kan bewerkt of verwerkt worden met behulp van één of meerdere \texttt{AudioProcessors}. Een \texttt{AudioProcessor} is een interface met de methodes \texttt{process} en \texttt{processingFinished}. De \texttt{process} methode heeft als enige parameter een \texttt{AudioEvent}. Dit object bevat een audio blok, voorgesteld als array van floating-point getallen met waarden tussen -1.0 en 1.0. De grootte van dit blokje audio, en de mate van overlapping tussen de opeenvolgende blokjes audio is instelbaar. Verder bevat dit object nog andere metadata zoals onder meer een \textit{timestamp}.

Afhankelijk van de implementatie van de \texttt{process} methode kan de audiostroom op een bepaalde manier verwerkt, geanalyseerd of gewijzigd worden.

TarsosDSP bevat verder nog een groot aantal klassen met allerhande tools en audioverwerkings algoritmen. Het merendeel hiervan maakt gebruik van de zojuist beschreven processing pipeline. 

Een greep uit de features van TarsosDSP:
\begin{itemize}[noitemsep]
	\item Toevoegen van geluidseffecten (delay, flanger,...)
	\item Toevoegen van filters (low-pass, high-pass, band-pass,...)
	\item Conversie tussen verschillende formaten
	\item Toonhoogte detectie
	\item Wijzigen van de samplefrequentie
	\item FFT analyse
\end{itemize}

\subsection{Panako}

Panako is net zoals TarsosDSP een Java bibliotheek (door de zelfde auteur) ontwikkeld aan het IPEM. Panako bevat een arsenaal aan algoritmen voor het matchen of synchroniseren van audiofragmenten of audiostreams.  Deze bibliotheek wordt uitgebreid beschreven in artikel \cite{six2014panako}.

In dit onderzoek wordt vooral gebruik gemaakt van Panako's open-source implementatie van het accoustic fingerprinting algoritme. Dit algoritme wordt beschreven in de paper van Avery Li-Chun Wang\cite{Wang2003a}. Het algoritme uit de paper is verder uitgebreid zodat audio waarbij de toonhoogte verhoogd of verlaagd is, of audio die sneller of trager is afgespeeld, toch gedetecteerd kan worden.

Buiten het eigenlijke algoritme bevat de bibliotheek ook verschillende toepassingen die hiervan gebruik maken. Zo is het mogelijk om de fingerprints van een geluidsfragment te bekijken, matches tussen verschillende geluidsfragmenten te visualiseren, en grafisch te experimenteren met de verschillende parameters. Figuur \ref{fingerprint_visualiser} toont een screenshot van deze toepassing.

\begin{figure}[!h]
	\captionsetup{width=0.8\textwidth}
	\caption[Gebruikersinterface van Audacity]{De gebruikersinterface van Panako's NFFT Fingerprint Visualiser. Onderaan kunnen de parameters van het algoritme met behulp van sliders gewijzigd worden.}
	\centering
	\advance\parskip0.3cm
	\includegraphics[width=0.8\textwidth]{fingerprint_visualiser.png}
	\label{fingerprint_visualiser}
\end{figure}

Er is ook een applicatie beschikbaar om verschillende geluidsfragmenten te synchroniseren. Deze applicatie maakt behalve van het accoustic fingerprinting algoritme ook nog gebruik van het kruiscovariantie algoritme. 

Wanneer de latency tussen de verschillende audiofragmenten bepaald is, dan kan de applicatie een shell script genereren dat met behulp van \textit{FFmpeg} stukjes van de geluidsbestanden wegknipt of er stilte aan toevoegt. Het resultaat is dat na het uitvoeren van het script de geluidsbestanden gesynchroniseerd zijn.

\subsection{FFmpeg}

FFmpeg is een command-line multimedia framework dat gebruikt wordt voor encoderen, decoderen, multiplexen, demultiplexen, streamen en afspelen van audio en video. \cite{kollarconfiguration}

In dit onderzoek wordt FFmpeg voornamelijk gebruikt in scripts bij het geautomatiseerd genereren van testdata.

\subsection{SoX}

SoX is net zoals FFmeg een command-line tool voor audioverwerking. Buiten de mogelijkheid om audiobestanden te converteren laat SoX ook minder triviale operaties toe. Zo is het onder meer mogelijk om het volume aan te passen, effecten toe te voegen, de bestanden bij te knippen of gegenereerde geluiden in een audiobestand te mixen.
\cite{barras2012sox}

In dit onderzoek wordt SoX ook gebruikt in scripts bij het manipuleren van de testdata.

\subsection{Sonic Visualiser}

Sonic Visualiser is een gebruiksvriendelijke desktopapplicatie voor de analyse, visualisatie van audiobestanden. Sonic Visualiser laat toe om audiobestanden vanuit verschillende perspectieven te analyseren, zo kan zowel de waveform als het spectrogram van een audiobestand gevisualiseerd worden. Sonic Visualiser is uitbreidbaar met plug-ins in het Vamp formaat. \cite{cannam2010sonic}

\begin{figure}[!h]
	\captionsetup{width=0.8\textwidth}
	\caption[Gebruikersinterface van Sonic Visualiser]{De gebruikersinterface van Sonic Visualiser}
	\advance\parskip0.3cm
	\centering
	\includegraphics[width=0.8\textwidth]{sonicvisualiser.png}
\end{figure}

Sonic Visualiser werd in dit onderzoek vooral gebruikt om handmatig de latency tussen verschillende audiofragmenten te bepalen. Ook heeft de applicatie dienst gedaan als educatieve tool om verschillende audioverwerkingsalgoritmen visueel voor te stellen.
\vspace{0.5cm}

\subsection{Audacity}

Audacity is een open-source desktopapplicatie voor het bewerken, opnemen en converteren van audio. Met Audacity is het ook mogelijk om tal van effecten en filters aan audio toe te voegen.\cite{audacity2016}

\begin{figure}[!h]
	\captionsetup{width=0.8\textwidth}
	\caption[Gebruikersinterface van Audacity]{De gebruikersinterface van Audacity}
	\centering
	\advance\parskip0.3cm
	\includegraphics[width=0.8\textwidth]{audacity.png}
	\label{screenshot-audacity}
\end{figure}

Alle opnames en handmatige bewerkingen op audiobestanden in dit onderzoek zijn uitgevoerd met behulp van Audacity. 

Figuur \ref{screenshot-audacity} toont de gebruikersinterface van dit programma.

\subsection{Max/MSP}

Max/MSP is een visuele programmeertaal voor muziek en multimedia. Het is een systeem waarbij modules met elkaar verbonden kunnen worden om zo complexere systemen op te bouwen. Max/MSP beschikt ook over een API waarmee in Java of C nieuwe modules ontwikkeld kunnen worden. \cite{cycling2016}

\begin{figure}[!h]
	\captionsetup{width=0.8\textwidth}
	\caption[Gebruikersinterface van MAX/MSP]{De gebruikersinterface van Max/MSP: een \textit{patch panel} met daarop enkele modules die samen een complexere toepassing vormen.}
	\centering
	\advance\parskip0.3cm
	\includegraphics[width=0.8\textwidth]{maxmsp.png}
	\label{screenshot-max}
\end{figure}

Met Max/MSP is het mogelijk om realtime audio verwerken, daarom zal deze toepassing gebruikt worden voor het ontwikkelen van de gebruikersinterface. 

Figuur \ref{screenshot-max} toont hoe een eenvoudige Max/MSP patch er grafisch uitziet.

\subsection{Teensy}

De Teensy is een kleine microcontroller die via USB geprogrammeerd kan worden. De Teensy is compatibel met de Arduino software en is hierdoor zeer gebruiksvriendelijk. \cite{teensy2016}

De sensoren die gebruikt worden bij de experimenten van het IPEM zijn meestal aangesloten op Teensy microcontrollers.
Om de synchronisatiealgoritmes en het bijhorende systeem in een representatieve situatie te testen zal daarom ook gebruik gemaakt worden van een Teensy microcontroller.

\begin{figure}[!h]
	\captionsetup{width=0.7\textwidth}
	\caption[Teensy microcontroller]{De Teensy microcontroller verbonden met een infraroodsensor en microfoon op een breadboard.}
	\centering
	\advance\parskip0.3cm
	\includegraphics[width=0.4\textwidth]{teensy.jpg}
	\label{teensy-pic}
\end{figure}

In hoofdstuk \ref{evaluatie} wordt de testen die met de Teensy microcontroller zijn uitgevoerd meer in detail besproken. Figuur \ref{teensy-pic} toont een typische testopstelling waarbij twee sensoren zijn aangesloten op de microcontroller.

\section{Accoustic fingerprinting}

De Panako softwarebibliotheek bevat een implementatie van het accoustic fingerprinting algoritme. Om wijzigingen mogelijk te maken hebben is de code van het algoritme overgenomen in het project van dit onderzoek. De overgenomen code blijft wel nog steeds afhankelijk van enkele klassen uit het Panako project. 

\subsection{Optimalisaties}

Aan dit algoritme is één vereenvoudiging aangebracht. Het originele algoritme bevatte namelijk de mogelijkheid om alle offsets boven een bepaalde drempelwaarde te verwerken. Deze feature laat toe dat er meerdere matches kunnen gevonden worden binnen één uitvoering van het algoritme. Om dit te ondersteunen moeten alle matches echter wel één voor één worden vergeleken met de drempelwaarde. Omdat we in dit onderzoek enkel geïnteresseerd zijn in de beste offsetwaarde is dit overbodig. De beste offset en bijhorende fingerprints wordt apart bijgehouden. De naverwerking wordt hierdoor vermeden.

\subsection{Parameters en hun invloed op het algoritme}
\label{accoustic-fingerprinting-params}

De werking van dit algoritme is afhankelijk van een aantal parameters die een grote invloed kunnen hebben op de performantie en de nauwkeurigheid van het uiteindelijke resultaat. Daarom is het van belang om voor het uitvoeren van het algoritme de waarde van deze parameters te controleren. De optimale waarde van elke parameter is afhankelijk van verschillende factoren die van situatie tot situatie kunnen verschillen:

\begin{itemize}[noitemsep]
\item de vereiste nauwkeurigheid van het algoritme;
\item de vereiste performantie van het algoritme;
\item de mate waarin er omgevingsgeluid aanwezig is;
\item de opnamekwaliteit van het omgevingsgeluid.
\end{itemize}

De meeste parameters worden bijgehouden in een configuratiebestand waardoor ze ook na compilatie wijzigbaar zijn. Dit zijn de belangrijkste parameters uit het configuratiebestand die invloed hebben op het algoritme:

\begin{description}
\item\texttt{SAMPLE\_RATE} \hfill \\
Deze parameter bepaalt de standaard samplefrequentie die gebruikt wordt tijdens het synchronisatieproces. Het verhogen van deze parameter zorgt voor een tragere verwerking maar een betere nauwkeurigheid. Afhankelijk van op welke manier de synchronisatie wordt opgestart (via Max of met een \texttt{AudioDispatcher}) worden de binnenkomende streams geresamplet of wordt al een correcte samplefrequentie verondersteld.

\item\texttt{NFFT\_BUFFER\_SIZE}\footnote{De letter N in NFFT heeft geen noemenswaardige betekenis. De naam is overgenomen van de gelijknamige parameter uit de Panako bibliotheek.} \hfill \\
Dit is de grootte van de verschuivende buffer die gebruikt wordt in het FFT algoritme. Deze parameter is cruciaal aangezien de frequentiesterktes op een bepaalde plaats op de tijd\-as worden berekend per buffer. Deze parameter wordt uitgedrukt in aantal samples.
\item\texttt{NFFT\_STEP\_SIZE} \hfill \\
Dit is het aantal samples elke verschuiving in het FFT algoritme. De stepsize beïnvloedt rechtstreeks de nauwkeurigheid van het accoustic fingerprinting algoritme. Wanneer deze parameter is ingesteld op 128 samples en een samplefrequentie van 8000hz dan is de maximale nauwkeurigheid $128/8000Hz = 0.016s = 16ms$.
\item\texttt{MIN\_ALIGNED\_MATCHES} \hfill \\
Een match tussen twee audiofragmenten wordt pas als geldig beschouwd wanneer er een bepaald aantal fingerprint matches met dezelfde offset gevonden zijn. Dit aantal wordt ingesteld met deze parameter.
\item\texttt{NFFT\_MAX\_FINGERPRINTS\_PER\_EVENT\_POINT} \hfill \\
Deze parameter bepaalt het maximum aantal fingerprints waaraan een event point (een punt op het spectrogram) kan deelnemen. Hoe hoger dit maximum hoe vlugger er matches kunnen gevonden worden. Bij een hoge waarde moeten meer berekeningen worden uitgevoerd, dit heeft invloed op de performantie.
\item\texttt{NFFT\_EVENT\_POINT\_MIN\_DISTANCE} \hfill \\
Dit is de minimale afstand tussen twee event points op het spectrogram die samen een fingerprint kunnen vormen. 

\end{description}

Verder maakt het algoritme nog gebruik van twee hardgecodeerde parameters die niet instelbaar zijn via het configuratiebestand: \texttt{MIN\_FREQUENCY} en \texttt{MAX\_FREQUENCY}. Deze parameters bepalen binnen welke frequentiebereik er naar fingerprints gezocht worden. De waarden waarop deze ingesteld staan bevinden zich op de rand van de frequenties die door muziek of stemgeluid geproduceerd worden.

\subsection{Optimale instellingen}
\label{optimal-accoustic-fingerprinting}

Het bepalen van de optimale waarden voor de parameters is geen exacte wetenschap maar eerder een probleem dat proefondervindelijk moet worden aangepakt.

Bij het accoustic fingerprinting algoritme is er een groot verschil tussen de meest ``elegante'' parameterwaarden en de in de praktisch presterende waarden. Dit verschil zal duidelijk worden in volgende opsomming waarin elke parameter zal worden besproken.


\begin{description}
	\item\texttt{SAMPLE\_RATE} \hfill \\
	Bij deze parameter is het van belang om een goede balans te vinden zodat de geluidskwaliteit aanvaardbaar blijft zonder een hypotheek te plaatsen op de performantie van het algoritme. De praktijk heeft uitgewezen dat bij een samplefrequentie van $ 8000Hz $ het algoritme goed presteert. Deze waarde wordt bevestigd in artikel \cite{six2015multimodal}.
	\item\texttt{NFFT\_BUFFER\_SIZE} en \texttt{NFFT\_STEP\_SIZE} \hfill \\
	De ingesteld waarden van de samplefrequentie, buffergrootte en stapgrootte zijn afhankelijk van elkaar. Om een goede werking van het FFT algoritme te garanderen bij een samplefrequentie van $ 8000Hz $ worden de buffergrootte en stapgrootte respectievelijk ingesteld op 512 en 128 (of 256) samples. Deze waarden worden eveneens vermeld in artikel \cite{six2015multimodal}.
	\item\texttt{MIN\_ALIGNED\_MATCHES} \hfill \\
	In een toepassing zoals het detecteren van liedjes ten opzichte van een database is het secuur instellen van deze parameter erg belangrijk. Deze parameter heeft namelijk een grote invloed op het voorkomen van \textit{false positives} of \textit{false negatives}. 
	
	Bij het synchroniseren van streams is de situatie echter helemaal anders. Het binnenkrijgen van een false positive (=foute latency) is veel minder erg dan het helemaal niet binnenkrijgen van (mogelijk correcte) resultaten. Aangezien het algoritme per buffer enkel het beste resultaat teruggeeft is de kans dat bij geluidsfragmenten van behoorlijke kwaliteit eenzelfde foute latency meer voorkomt dan de correcte latency zéér klein.
	
	Bij geluidsfragmenten van mindere kwaliteit kan het gebeuren dat er toch foute resultaten door de mazen van het net glippen. Om dit te vermijden is het mogelijk om nog een extra filtering toe te passen. In deze extra stap worden eventuele uitschieters geëlimineerd.
	
	Bovenstaande argumenten stellen duidelijk dat het beter is om deze parameter een lage waarde te geven. In deze toepassing is gekozen voor de waarde 2 in plaats van het absolute minimum 1: hierdoor wordt pure willekeur bij het matchen van audiofragmenten van extreem slechte kwaliteit vermeden.
	
	\item\texttt{NFFT\_MAX\_FINGERPRINTS\_PER\_EVENT\_POINT} \hfill \\
	In Panako is het standaard dat een event point uitmaakt van maximaal 2 fingerprints. Het hanteren van deze waarde leidt ertoe dat het matchen van audiofragmenten van behoorlijke kwaliteit zeer snel kan worden uitgevoerd. 
	
	In deze toepassing is het aantal te vergelijken audiofragmenten meestal erg beperkt. Ook is de kwaliteit van deze fragmenten vaak van ondermaats (bv. de opnames op microcontrollers). Daarom is het in dit geval een goed idee om de waarde van deze parameter zéér hoog in te stellen. Hierdoor verhoogt de kans sterk dat er bij zeer slechte audiofragmenten toch enkele overeenkomende fingerprints gevonden worden. Testen hebben uitgewezen dat de negatieve invloed op de performantie beperkt blijft en dat de resultaten sterk verbeteren. 
	
	Bij het zoeken naar matches tussen geluidsopnames opgenomen op microcontrollers heeft de praktijk uitgewezen dat het toelaten van maximaal 50 fingerprints per event point degelijke resultaten oplevert.
	%TODO: refereren naar testresultaen
	
	\item\texttt{NFFT\_EVENT\_POINT\_MIN\_DISTANCE} \hfill \\
	In tegenstelling tot vorige parameter zorgt het verhogen van deze waarde ervoor dat er minder fingerprints worden gecreëerd. De argumenten die bij vorige parameter zijn aangehaald gelden bijgevolg in omgekeerde zin ook voor deze parameter. Hoewel de Panako standaard 600 is leveren waardes rond het getal 20 in deze toepassing de beste resultaten zonder de performantie sterk te beperken.
	
	
\end{description}


\section{Kruiscovariantie}

De Panako bibliotheek bevatte bij aanvang van dit onderzoek ook al een implementatie van het kruiscovariantie algoritme. In tegenstelling tot het accoustic fingerprinting algoritme was het echter minder grondig afgewerkt. Om degelijke resultaten te garanderen was het noodzakelijk om enkele anomalieën in de geleverde resultaten te analyseren en de oorzaak hiervan op te lossen.

Net zoals het accoustic fingerprinting algoritme is de code overgenomen in het project van dit onderzoek. De code is echter niet meer afhankelijk van de Panako bibliotheek.

\subsection{Integratie met accoustic fingerprinting}
\label{integratie-accoustic-fingerprinting}

In sectie \ref{toepasbaarheid} is vermeld dat de latency eerst ruw berekend wordt met accoustic fingerprinting. Dit resultaat wordt vervolgens verfijnd door de kruiscovariantie tussen de signalen te berekenen. De werking hiervan zal verduidelijkt worden met behulp van een voorbeeld.

Onderstel twee geluidsopnames: de tweede opname heeft een vertraging van 90 ms ten opzichte van de eerste referentieopname. Het uitvoeren van het accoustic fingerprinting algoritme met standaard parameters (tot op $ 32 ms $ nauwkeurig) kan twee resultaten opleveren: $ 64 ms $ of $ 96 ms $.\footnote{Mogelijke resultaten: $ 2 \cdot 32 ms = 64 ms$ of $ 3 \cdot 32 ms = 96 ms $. De absolute fout is  met een waarde van $ 6 ms $ bij een latency van $ 96 ms $ het kleinst in vergelijking met $ 26 ms $ bij een latency van $ 64 ms $. Daarom is de kans het grootst dat $ 96 ms $ als resultaat zal worden teruggegeven.} Bij het eerste resultaat wordt de werkelijke latency onderschat. De tweede latency overschat deze waarde. Na het berekenen van de kruiscovariantie is dit van belang om het resultaat correct te verfijnen. In figuur \ref{crosscovariance1} worden de situatie visueel verduidelijkt.

\begin{figure}[h!]
	\captionsetup{width=0.7\textwidth}
	\caption[Kruiscovariantie audiofragmenten]{Twee audiofragmenten: het tweede audiofragment heeft een latency van 90 milliseconden ten opzichte van het eerste.}
	\begin{center}
		\advance\parskip0.3cm
		\begin{tikzpicture}[scale=1]
\begin{axis}[
xlabel={Tijd (milliseconden)},
xmin=0,xmax=500,
ymin=0,ymax=1,
legend style={
	cells={anchor=west},
	legend pos=outer north east,
},
yticklabels={,,},
xticklabel style={grid=major},
extra x ticks ={64,90,96},
extra x tick labels={,,,},
extra tick style={grid=major, grid style={dotted}},
hide y axis
]

\addplot[thick,black] graphics[xmin=0,ymin=0.5,xmax=500,ymax=0.8] {reference.png};

\addplot[thick,black] graphics[xmin=0,ymin=0.0,xmax=500,ymax=0.3] {other.png};


after end axis/.code={

	
	\draw[black,<->] (axis cs:0,0.42) -- (axis cs:96,0.42)	node [pos=0.5,above,font=\scriptsize] {$96ms$};
	
	\draw[black,<->] (axis cs:0,0.32) -- (axis cs:90,0.32)	node [pos=0.5,above,font=\scriptsize] {$90ms$};
	
	\draw[black,<->] (axis cs:0,0.22) -- (axis cs:64,0.22)	node [pos=0.5,above,font=\scriptsize] {$64ms$};
}]

\node at (axis cs:250,0.82) [] {Referentie};
\node at (axis cs:250,0.35) [] {Andere};

\end{axis}

\end{tikzpicture}
	\end{center}
	\label{crosscovariance1}
\end{figure}

Na het uitvoeren van het accoustic fingerprinting algoritme wordt er van beide fragmenten een blokje audio weggeknipt afhankelijk van de ruw bepaalde latency. Na deze stap is de latency van het resterende audiofragment zeker minder dan de minimale nauwkeurigheid van het accoustic fingerprinting algoritme. Vervolgens wordt het begin van elk audiofragment gekopieerd naar een buffer van een honderdtal samples. Daaropvolgend wordt de latency tussen deze twee buffers berekend met de kruiscovariantie methode. Deze latency is echter nog niet de latency waarnaar er uiteindelijk gezocht wordt. Er is nog een extra stap vereist waarbij de latency van het accoustic fingerprinting algoritme gecombineerd wordt met de latency van het kruiscovariantie algoritme. Figuur \ref{crosscovariance2} toont de buffers waarop het algoritme kan worden uitgevoerd. Elke buffer bevat 512 samples ($ 64 ms $ bij een samplefrequentie van $ 8000 Hz $).

\begin{figure}[h!]
	\captionsetup{width=0.7\textwidth}
	\caption[Kruiscovariantie buffers]{Kruiscovariantie buffers na het wegknippen van de $ 96 ms $ latency bepaald door het accoustic fingerprinting algoritme. Hier heeft de referentie audio $ 6 ms $ latency ten opzichte van het andere audiofragment.   }
	\begin{center}
		\advance\parskip0.3cm
		\begin{tikzpicture}[scale=1]
\begin{axis}[
xlabel={Tijd (milliseconden)},
xmin=0,xmax=64,
ymin=0,ymax=1,
legend style={
	cells={anchor=west},
	legend pos=outer north east,
},
yticklabels={,,},
xticklabel style={grid=major},
%extra x ticks ={64,90,96},
extra x tick labels={,,,},
extra tick style={grid=major, grid style={dotted}},
hide y axis,
width = 15cm,
height = 7cm
]

\addplot[thick,black] graphics[xmin=0,ymin=0.6,xmax=64,ymax=0.9] {refcut.png};

\addplot[thick,black] graphics[xmin=0,ymin=0.1,xmax=64,ymax=0.4] {othercut.png};


after end axis/.code={
	\draw[black,<->] (axis cs:0,0.85) -- (axis cs:6,0.85)	node [pos=0.5,above,font=\scriptsize] {$6ms$};
}]

\node at (axis cs:32,0.85) [] {Referentie};
\node at (axis cs:32,0.37) [] {Andere};

\end{axis}

\end{tikzpicture}
	\end{center}
	\label{crosscovariance2}
\end{figure}

Intuïtief lijkt het logisch dat deze latencies enkel bij elkaar moeten worden opgeteld om de verfijnde latency te bekomen. Toch is dit niet helemaal correct. Wanneer het accoustic fingerprinting algoritme de werkelijke latency heeft onderschat levert deze optelling een correct resultaat. Bij een overschatting is dit echter niet het geval. Dit probleem is duidelijk zichtbaar in figuur \ref{crosscovariance2}: na het wegknippen van de ruwe latency ($96 ms$) is de latency omgedraaid: de referentie audio heeft $ 6 ms $ latency ten opzichte van het andere audiofragment. 

Aangezien het kruiscovariantie algoritme de latency zoekt van de ``andere'' buffer ten opzichte van de referentiebuffer zal het resultaat niet $6 ms$ maar $ 58 ms $ zijn. Dit komt doordat het algoritme de ``andere'' buffer cyclisch (maar in de verkeerde zin) verschuift. Een verschuiving van $ 58 ms $ is bijgevolg de meest optimale verschuiving.

Om in een dergelijke situatie de verfijnde latency te berekenen moet men de latency verkregen door het kruiscovariantie algoritme ($ 58 ms$) aftrekken van de lengte van de gebruikte buffer ($ 64 ms $). De verfijnde latency bekomt men vervolgens door dit resultaat ($ 6 ms $) af te trekken van de latency van het fingerprinting  algoritme ($ 96 ms - 6 ms = 90 ms $).

%Eventueel nog schrijven hoe bepaald wordt of het resultaat onderschat of overschat is.

\subsection{Optimalisaties}

\subsubsection{Bugfixes}

In de originele versie van het algoritme werd er geen rekening gehouden met het feit dat het accoustic fingerprinting algoritme het de werkelijke latency kan overschatten of onderschatten. De latency bepaalt door het kruiscovariantie algoritme werd telkens opgeteld bij de ruwe latency bepaald door het accoustic fingerprinting algoritme. In 50\% van de gevallen leidde dit probleem tot foutieve resultaten.

\subsubsection{Meerdere malen uitvoeren van het algoritme}

In tegenstelling tot accoustic fingerprinting is het bepalen van de latency met kruiscovariantie veel meer foutgevoelig. Bij opnames van slechte kwaliteit worden vaak foute resultaten teruggegeven. 

Hoewel bij accoustic fingerprinting mogelijk storende factoren (een zoemende toon, geruis,...) zichtbaar zijn in het spectrogram zullen er naar alle waarschijnlijkheid nog steeds voldoende fingerprints gegenereerd worden op basis van geluiden die wel in beide opnames voorkomen. 

\begin{figure}[h!]
	\captionsetup{width=0.7\textwidth}
	\caption[Zoemtoon van $50 Hz$]{Twee gelijke audiofragmenten. Aan het tweede audiofragment is een achtergrondtoon van $50 Hz$ toegevoegd.}
	\begin{center}
		\advance\parskip0.3cm
		\begin{tikzpicture}[scale=1]
\begin{axis}[
xlabel={Tijd (milliseconden)},
xmin=0,xmax=140,
ymin=0,ymax=1,
legend style={
	cells={anchor=west},
	legend pos=outer north east,
},
yticklabels={,,},
xticklabel style={grid=none},
extra x tick labels={,,,},
extra tick style={grid=major, grid style={dotted}},
hide y axis,
width = 15cm,
height = 6cm
]

\addplot[thick,black] graphics[xmin=0,ymin=0.6,xmax=140,ymax=0.9] {zoemtoon-origineel.png};

\addplot[thick,black] graphics[xmin=0,ymin=0.1,xmax=140,ymax=0.4] {zoemtoon.png};


after end axis/.code={
%	\draw[black,<->] (axis cs:0,0.85) -- (axis cs:6,0.85)	node [pos=0.5,above,font=\scriptsize] {$6ms$};
}]

\node at (axis cs:70,0.88) [] {Origineel fragment};
\node at (axis cs:70,0.40) [] {Fragment met $50 Hz$ toon};

\end{axis}

\end{tikzpicture}
	\end{center}
	\label{zoemtoon}
\end{figure}

Dergelijke storende elementen hebben veel meer invloed op het kruiscovariantie algoritme. Een ongewenste zoemende bastoon zorgt voor een grote verandering van de geluidsgolf van het audiofragment. Aangezien de kruiscovariantie rechtstreeks tussen twee geluidsgolven berekent wordt kan dit serieuze gevolgen hebben. In figuur \ref{zoemtoon} wordt dit probleem visueel verduidelijkt.

De invloed van storende factoren zoals zoemende tonen en ruis is gemakkelijk te beperken. Aangezien het kruiscovariantie algoritme over een zéér klein deeltje van het audiofragment wordt uitgevoerd is het zeer eenvoudig om het algoritme tientallen keren, maar telkens op een andere plaats, uit te voeren. Na elke iteratie wordt de latency in het geheugen opgeslagen. Ten slotte wordt er gezocht naar de latency die het meeste voorkomt. Die latency wordt gebruikt om het resultaat van het accoustic fingerprinting algoritme te verfijnen.

\subsection{Parameters en hun invloed op het algoritme}

De parameters van het kruiscovariantie algoritme zijn net zoals die van het accoustic fingerprinting algoritme instelbaar via het configuratiebestand.

\begin{description}	
	\item\texttt{NFFT\_BUFFER\_SIZE} \hfill \\
	Hoewel het FFT algoritme eigenlijk niets te maken heeft met het berekenen van de kruiscovariantie speelt deze parameter bij dit algoritme toch een belangrijk rol. Deze parameter bepaalt namelijk de grootte van de buffers waartussen de kruiscovariantie berekent wordt. Om een goede werking van het kruiscovariantie algoritme te verzekeren is het een vereiste dat de gebruikte buffers groter zijn dan de minimale nauwkeurigheid van het accoustic fingerprinting algoritme (anders is het mogelijk dat na het knippen de audiofragmenten er geen gelijkenissen te vinden zijn tussen beide buffers). Aangezien deze nauwkeurigheid bepaald wordt de \texttt{NFFT\_STEP\_SIZE} parameter en de \texttt{NFFT\_BUFFER\_SIZE} hier altijd een veelvoud van is, is het gemakkelijk om deze parameter ook voor het kruiscovariantie algoritme te gebruiken.
	
	\item\texttt{CROSS\_COVARIANCE\_NUMBER\_OF\_TESTS} \hfill \\
	Deze parameter bepaalt het aantal keren dat het kruiscovariantie algoritme per streambuffer (zie \ref{streambuffers}) zal worden uitgevoerd. De meest voorkomende latency wordt als resultaat teruggegeven.
	
	\item\texttt{CROSS\_COVARIANCE\_THRESHOLD} \hfill \\
	Deze parameter bepaalt het minimum aantal keer dat een latency moet voorkomen om als geldig beschouwt te mogen worden.	Stel dat het algoritme 50 keer wordt uitgevoerd maar elke keer een verschillend resultaat teruggeeft, dan zal er geen latency worden teruggegeven indien deze parameter staat ingesteld op een waarde groter dan 1.
	
\end{description}

\subsection{Optimale instellingen}
\begin{description}	
	\item\texttt{NFFT\_BUFFER\_SIZE} \hfill \\
	De optimale waarde van deze parameter werd in sectie \ref{optimal-accoustic-fingerprinting} besproken.
	
	\item\texttt{CROSS\_COVARIANCE\_NUMBER\_OF\_TESTS} \hfill \\
	Bij audiofragmenten van goede kwaliteit is het niet noodzakelijk om deze parameter een hoge waarde te geven. Als het algoritme 5 à 10 keer wordt uitgevoerd zal de correcte latency hoogst waarschijnlijk al verschillende malen gevonden zijn. 
	
	Bij audiofragmenten van slechte kwaliteit is het van belang om deze parameter zo hoog mogelijk in te stellen. Aangezien het kruiscovariantie algoritme vrij intensief is hangt de waarde wat af van de beschikbare rekenkracht. De praktijk heeft uitgewezen dat de waarde 50 een goed evenwicht biedt tussen correctheid en performantie.
	
	Dit aantal heeft als bovengrens ($ N_B $) het aantal keren dat een kruiscovariantie buffer (bepaald door \texttt{NFFT\_BUFFER\_SIZE}) past in een streambuffer (zie \ref{streambuffers}) waarin de streams worden ingelezen (bepaald door \texttt{SLICE\_SIZE\_S}). Dit kan worden berekend met volgende formule:
	\begin{equation}
		N_B = \frac{\texttt{SLICE\_SIZE\_S}}{\texttt{NFFT\_BUFFER\_SIZE} \cdot \texttt{SAMPLE\_RATE}}
	\end{equation}
	
	\item\texttt{CROSS\_COVARIANCE\_THRESHOLD} \hfill \\
	Het hoog instellen van deze waarde heeft enkel nut als het van belang is dat het kruiscovariantie algoritme zeker een correct resultaat teruggeeft. In de huidige implementatie wordt het ruwe resultaat gebruikt als6 het aantal gelijke waarden niet boven deze drempel komt. Daarom wordt deze waarde meestal op 1 ingesteld. De best mogelijke verfijning van het resultaat wordt dan toegepast, ongeacht het aantal keer dat het resultaat voorkomt. 
\end{description}

\section{Filteren van de resultaten}
\label{filtering}

Het is moeilijk om met zekerheid te bepalen of een bepaalde latency correct is. Toch kan er een soort van statistische optimalisatie worden uitgevoerd op de opeenvolgende latencies. De kans is namelijk klein dat de werkelijke latency verandert en na een korte tijd terugkeert naar de originele waarde.\footnote{Theoretisch is een dergelijke situatie toch mogelijk: In dit voorbeeld wordt de latency bepaalt tussen de audiostreams ($ 8000 Hz$ samplefrequentie) van twee Teensy microcontrollers. Wanneer er 50 samples gedropt worden van de referentie audiostream zorgt dit voor een latencyverhoging van de andere audiostream van $6 ms$. Indien er enkele seconden later 50 samples gedropt worden van de andere audiostream dan leidt dit tot een vermindering van de latency van $ 6 ms $. Visueel is dit een piek in de opeenvolgende latencies.} Ook is de kans klein dat een foute latency toevallig verschillende malen na elkaar gedetecteerd wordt.

Deze eigenschappen laten toe om toch een bepaalde vorm van ``foutherstelling'' te implementeren. Fouten kunnen namelijk weggefilterd worden door eventuele pieken in opeenvolgende latencies af te vlakken.

\subsection{Werking}

Het afvlakken van pieken kan in realtime geïmplementeerd worden door opnieuw gebruik te maken van verschillende \textit{sliding windows}. Per audiostream wordt een buffer bijgehouden met daarin de opeenvolgende latencies ten opzichte van de referentie audiostream. Wanneer een buffer haar maximumcapaciteit heeft bereikt gaat het toevoegen van een nieuwe latency gepaard met het verwijderen van de oudste latency.

Het afvlakken van pieken wordt verwezenlijkt door in plaats van de meest recente latency het gemiddelde of de mediaan van de buffer te gebruiken. De mate waarin pieken worden afgevlakt hangt af van de grootte van de buffer en de gebruikte methode (gemiddelde of mediaan).

\subsection{Voorbeelden}

De effectiviteit van de verschillende parameters (buffergrootte en filtermethode) zal in dit gedeelte worden weergegeven aan de hand van enkele voorbeelden. Er zullen verschillende soorten filters worden toegepast op een verloop van 35 latencies. In het ongefilterde verloop (zie figuur \ref{latencydata}) komen twee grote en drie kleine pieken voor. Na het vijftiende resultaat doet er zich een blijvende verhoging van de latency voor.

\begin{figure}[h!]
	\captionsetup{width=0.7\textwidth}
	\caption[Het ongefilterde verloop van de latency]{Grafische weergave van het ongefilterde verloop van de latency.}
	\begin{center}
		\advance\parskip0.3cm
		\input{figuren/latencydata}
	\end{center}
	\label{latencydata}
\end{figure}

\subsubsection{Moving average filter}

De eerste twee voorbeelden tonen het verschil tussen de ongefilterde latencies en de latencies die gefilterd zijn door het gemiddelde van een buffer te nemen. Bij het eerste voorbeeld bevat de buffer maximum 3 latencies, bij het tweede voorbeeld kan de buffer maximum 5 latencies bevatten.

\begin{figure}[!tbph]
	\centering
	\subfloat[Buffergrootte 3]{\begin{tikzpicture}[scale=0.9]
\begin{axis}[
ylabel={Latency (ms)},
xtick ={0,5,10,15,...,35},
ytick ={60,80,...,180},
ymin = 60,
ymax = 180,
height = 5cm,
width = 8cm
]
\addplot [mark=none] table {avg3data.dat};
\end{axis}
\end{tikzpicture}
}
	\hfill
	\subfloat[Buffergrootte 5]{\input{figuren/avg5data}}
	\captionsetup{width=0.7\textwidth}
	\caption{Het verloop van de latencies na het toepassen van een moving average filter.}
\end{figure}

Ondanks het toepassen van de filter blijven de grootste pieken nog steeds zichtbaar. De oorzaak hiervan is dat deze pieken door hun grootte toch sterk doorwegen op het gemiddelde. Om er voor te zorgen dat grote pieken nog meer worden weggefilterd zou de maximumgrootte moeten worden verhoogd. In \ref{filter-gevolgen} zal duidelijk worden waarom dit toch geen goed idee is.

In de gefilterde resultaten valt ook op dat de hellingshoek van de latencyverhoging vermindert naarmate een grotere buffer gehanteerd wordt. Dit is geen goede eigenschap: het duurt namelijk veel langer tot er na een wijziging terug een stabiele latency bereikt wordt.

\subsubsection{Moving median filter}

In de volgende voorbeelden zal de mediaan van elke buffer berekent worden. Om de resultaten goed met elkaar te kunnen vergelijken worden dezelfde buffergroottes gehanteerd. Het is aangeraden om een oneven getal als buffergrootte te nemen. In de huidige implementatie zal namelijk van het overblijvende paar resultaten het eerste element genomen worden in plaats van het gemiddelde ervan te berekenen.

\begin{figure}[!tbph]
	\centering
	\subfloat[Buffergrootte 3]{\input{figuren/median3data}}
	\hfill
	\subfloat[Buffergrootte 5]{\input{figuren/median5data}}
	\captionsetup{width=0.7\textwidth}
	\caption{Het verloop van de latencies na het toepassen van een moving median filter.}
\end{figure}

De grafieken tonen een duidelijk verschil tussen de moving median filter en de moving average filter. De pieken veroorzaakt door 1 (eventueel) foute latency worden bij de moving median filter onmiddellijk weggefilterd. In de grafiek is er geen enkel spoor meer van te vinden. De piek veroorzaakt door 2 (eventueel) foute latencies wordt enkel volledig weggefilterd wanneer buffergrootte 5 gehanteerd wordt.

De wijziging van de latency wordt niet afgevlakt zoals bij de moving average filter. De hellingsgraad blijft onaangetast.

\subsection{Parameters}

\begin{description}
	\item\texttt{LATENCY\_FILTER\_TYPE} \hfill \\
	Deze parameter bepaalt welke soort filter zal worden toegepast op de binnenkomende latencies. Mogelijke waarden zijn: \texttt{average}, \texttt{median} en \texttt{none}.
	
	\item\texttt{LATENCY\_FILTER\_BUFFER\_SIZE} \hfill \\
	Dit is de grootte van de buffer waarin de meest recente latencies zullen worden opgeslagen. Zoals al is vermeld  is het aangeraden om als waarde een oneven getal te kiezen.

\end{description}

\subsection{Gevolgen}
\label{filter-gevolgen}

Het filteren van de latency heeft een negatief effect op de snelheid waarmee wijzigingen van de latency gedetecteerd kunnen worden. Zowel de grootte van de latency filter buffer als de grootte van de streambuffers (zie \ref{streambuffers}) hebben invloed op de snelheid waarmee nieuwe latencies gedetecteerd zullen worden. De grootte van de streambuffers (parameter \texttt{SLICE\_SIZE\_S}) bepaalt namelijk met welk interval nieuwe latencies binnenkomen. 

Een verhoging van de latency zal bij de moving average filter een onmiddellijke maar beperkte invloed hebben op het gefilterde resultaat. De tijd die nodig is om tot een stabiel resultaat ($ T_a $) te komen kan berekent worden met volgende formule:

\begin{equation}
	T_a = (\texttt{SLICE\_SIZE\_S}) \cdot (\texttt{LATENCY\_FILTER\_BUFFER\_SIZE})
\end{equation}

Bij de moving median filter wordt een wijziging van de latency gedetecteerd wanneer meer dan de helft van de buffer de gewijzigde latency bevat. De tijd tot wanneer een detectie plaatsvindt ($ T_m $) kan met volgende formule berekent worden:

\begin{equation}
T_m = \frac{(\texttt{SLICE\_SIZE\_S}) \cdot (\texttt{LATENCY\_FILTER\_BUFFER\_SIZE})}{2}
\end{equation}

\section{Ontwerp van de softwarebibliotheek}

De Java bibliotheek voor het synchroniseren van streams bestaat uit verschillende onderdelen onderverdeeld in \texttt{packages}. De klassen uit eenzelfde package vervullen samen één taak. In volgende paragraaf zullen de verschillende packages kort toegelicht worden. Vervolgens zal het ontwerp van de applicatie meer in detail besproken worden.

\begin{description}	
	\item\texttt{be.signalsync.stream} \hfill \\
	Deze package bevat klassen die verantwoordelijk zijn voor het abstract voorstellen en verwerken van streams. Streams kunnen namelijk via verschillende kanalen worden ingelezen. Met behulp van deze klassen wordt een abstractere verwerking mogelijk.
	
	\item\texttt{be.signalsync.slicer} \hfill \\
	De klassen uit deze package zorgen voor het bufferen van de binnenkomende streams (zie sectie \ref{streambuffers}) zodat het bepalen van de latency in realtime mogelijk wordt. 

	\item\texttt{be.signalsync.syncstrategy} \hfill \\
	Deze package bevat de implementatie van de synchronisatiealgoritmen. Deze algoritmen worden aangeroepen van uit het \texttt{be.signalsync.sync} package. Welk algoritme precies gebruikt wordt kan worden ingesteld via het configuratiebestand.
	
	\item\texttt{be.signalsync.datafilters} \hfill \\
	Deze package bevat de implementaties van de verschillende soorten filters besproken in sectie \ref{filtering}.
	
	\item\texttt{be.signalsync.sync} \hfill \\
	Deze package roept klassen uit het \texttt{be.signalsync.slicer} aan om binnenkomende streams op te splitsen in buffers (alias slices). Vervolgens worden de algoritmes uit \texttt{be.signalsync.syncstrategy} gebruikt om de latency tussen de opeenvolgende buffers te bepalen. Met behulp van het package \texttt{be.signalsync.datafilters} worden eventuele pieken uit de resulterende latencies weggefilterd.

	\item\texttt{be.signalsync.msp} \hfill \\
	Deze package bevat de implementatie van enkele Max/MSP modules. Met behulp van deze modules is het mogelijk om de streams van microcontrollers in te lezen en de data ervan te synchroniseren.
	
\end{description}

\subsection{Streams}

\subsection{Slicen van streams}

\subsection{Oproepen van de algoritmen}

\section{Max/MSP modules}

%Gans dit onderzoek heeft uiteindelijk de bedoeling te resulteren in het ontwerp en implementatie van een module in MAX/MSP\footnote{Cycling ‘74 Max/MSP is een softwarepakket en een visuele programmeertaal waarmee audio, video en multimedia kan worden verwerkt met behulp van onafhankelijke modules. Deze modules kunnen met elkaar worden verbonden om zo complexe zaken te bereiken. Buiten de standaard meegeleverde modules is het ook mogelijk om zelf modules te schrijven.\cite{maxmsp2016}} die realtime synchronisatie mogelijk moet maken via een interface die bruikbaar is voor musicologen/onderzoekers met een beperkte informatica achtergrond.

%Deze module moet in staat zijn om verschillende datastromen als input binnen te krijgen, de synchronisatiebibliotheek aan te roepen, en de gesynchroniseerde datastromen als output terug te geven. Een andere Max module kan er dan voor zorgen dat deze data wordt weggeschreven naar een WAVE-bestand.